\documentclass{sig-alternate}
\usepackage{amsmath}
\usepackage{amsfonts}
\usepackage{amssymb}
\usepackage{times}
\usepackage{microtype}
% \usepackage{graphicx, subfigure, fink, grffile, placeins}
\usepackage{hyperref}
\usepackage{color}
\usepackage{url}
\usepackage{algorithm2e}
\usepackage{caption}
\usepackage{subcaption}


 \usepackage{upgreek}
% \input{dfn.tex}
\newcommand{\ericx}[1]{\textcolor{red}{\\ eric-comment: #1}}
\newcommand{\abhi}[1]{\textcolor{blue}{\\ abhi-comment: #1}}

\newcommand{\order}[1]{\textit{O}(#1)}
\newcommand{\comment}[1]{\textcolor{red}{[#1]}}

\title{Large scale structure aware community discovery in online forums}
\numberofauthors{6}
\author{
\alignauthor
Abhimanu Kumar \\
\email{abhimank@cs.cmu.edu}
\alignauthor
Shriphani Palakodety \\
\email{shriphanip@gmail.com}
\alignauthor
Chong Wang \\
\email{chongw@cs.cmu.edu}
\alignauthor
\and
Miaomiao Wen\\
\email{mwen@cs.cmu.edu}
\alignauthor
Carolyn P. Rose\\
\email{cprose@cs.cmu.edu}
\alignauthor
Eric P. Xing\\
\email{epxing@cs.cmu.edu}
\sharedaffiliation
\affaddr{School of Computer Science}  \\
\affaddr{Carnegie Mellon University}  \\
}



\newcommand{\fix}{\marginpar{FIX}}
\newcommand{\new}{\marginpar{NEW}}
%\newcommand{\comment}[1]{{\color{red}{#1}}}

%\nipsfinalcopy

\def\sharedaffiliation{%
\end{tabular}
\begin{tabular}{c}}

\begin{document}
\maketitle
\begin{abstract}
Online discussion forums are complex webs of overlapping communities. 
Users choose to participate in threads according to their interests, 
whether fleeting or more persistent. Thus, the network structure left 
behind as their behavior trace can be seen as encoding evidence of 
coordinated interests and values shared within sub-communities that 
members move among from time to time.  We propose a latent 
hierarchical model to incorporate this structure that leverages topic 
models (LDA) along with mixed membership stochastic block (MMSB) 
models. This makes the model complex and traditional sampling or 
variational based inference are very slow to converge on it. We overcome 
this by a proposing a new parallel inference scheme based on stochastic 
variational inference that is highly scalable. We evaluate our 
model on three large-scale datasets, which we refer to as 
Cancer-ThreadStarter (22K users and 14.4K threads), Cancer-NameMention(15.1K 
users and 12.4K threads) and StackOverFlow (1.19 million users and 
4.55 million threads). Qualitatively, we demonstrate that our 
model uncovers new communities that were hitherto invisible to a 
simpler MMSB or spectral clustering based methods. Quantitatively, we 
show that our model performs significantly better than MMSB, LDA, 
matrix factorization and a bag of words model in predicting 
user reply structure within threads. In addition, we demonstrate 
via rigorous synthetic data experiments that the proposed active sub-network 
discovery model is stable and recovers the original parameters of 
the experimental setup with high probability. We also provide speed 
comparisons of our model with varying degree of parallelism and data sizes.  

\end{abstract}


% NOTE: I make the argument that Poisson can be used to handle zero edges. and
% then I say that I use zero edges in Stack Overflow. I make sure to say that I am
% using equal proportion of those edges in training and test. Also I can take
% Prem's defense and say that I learn that param of weight.

\section{Introduction}
Online forums are a microcosm of communities where users' presentation
characteristics vary across different regions of the forum. Users participate in
a discussion or group activity by posting on a related thread; during one's
stay on a forum, a user participates in many different discussions and posts on
multiple threads. The thread-level presentation characteristics of a user are different
than the global presentation characteristics. A participating user gears his
responses to suit specific discussions on different threads. These thread based
interactions give rise to active sub-networks, within the global network of users,
that characterize the dynamics of interaction. Overlaying differential changes
in user interaction characteristics across these sub-networks provides
insights into user communities that exist in the forum. 

Analysing online social networks and user forums have been approached using
various perspectives such as network ~\cite{Shi:2000:NCI:351581.351611,
Shi00learningsegmentation} , probabilistic 
graphical models~\cite{ Airoldi:2008:MMS:1390681.1442798}, 
combined network \& text 
~\cite{Ho:2012:DHT:2187836.2187936,Nallapati:2008:JLT:1401890.1401957}. 
However none of these have taken
into account the dynamics of sub-networks and the related thread-based framework
within which forum discussions take place. Whereas
active sub-network modelling has been very useful to the research in
computational biology in recent years where it's been used to model sub-networks
of gene interactions~\cite{journals/ploscb/DeshpandeSVHM10,Lichtenstein:Charleston},
very few approaches using active sub-network have been proposed to model online
 user interactions. Taking into account sub-network interaction
dynamics is important to correctly model the user participant behavior. For
example, users post their
responses on discussion threads after reading through responses of
other users in the threads. The users possibly post multiple times on the thread
 as a form of reply to other posts in the thread. For analysing such
 interactions it becomes imperative that the structure of the conversation must also be taken
into account  besides the user interaction network and the
text posted. This enables us to gain deeper insights into user behavior in the
online community that was not possible earlier. 

The aim of this work is to provide novel insights into online forum discussions
and the related communities that emerge. Our proposed approach contrast existing techniques built on  
simple aggregation of the transient forum user networks into a single graph and user 
text across forums into a single document per user, which can cause severe loss of information about the forum. 
For example, we show that modeling the thread structure of the forum along with text and network 
leads to discovery of new communities that were overlooked by simple
aggregated-network analysis methods such as spectral learning, MMSB, etc. Our approach also 
provides better user-user link prediction results. 
%Though the  
%model is complex, it is worthwhile due to the better insights that it provides.
The proposed model incurs a heavy computational complexity due to a non-trivial network-text interplay at both latent variable and parameter levels, and we also present a highly scalable procedure based on stochastic variational inference that can handle millions of users with billions of threads in matter of hours. We believe our proposed approach is practical for analyzing real world forums, as we demonstrated in our experiments. Bellow, we highlight the specific challenges we addressed, and the main technical contributions we made.

\vspace*{-0.5\baselineskip}
\paragraph{Challenges} Modelling text, network and 
structure together such that it provides
quantifiable improvments over either of the three is a big challenge.
Moreover we model weighted edges instead of binary edges 
as is the case in traditional MMSB model. This adds to the 
complexity of the model and makes the single-processor inference challenging.
Another main challenge of this work has been the ability to model
active sub-networks in a large forum with millions of users and
threads. A social network spanning around millions of users and threads would 
be an ideal case to demonstrate the effectiveness of sub-network
modelling and thread-based structure.
To efficiently scale our model, we derive a scalable inference based on
stochastic variational inference (SVI) with sub-sampling~\cite{Hoffman:2013:SVI} 
that has the capacity to deal with such massive scale data and parameter space.
 The scalability of the SVI
with sub-sampling is further boosted by employing Poisson distribution 
to model edge weights of
the network. A Poisson based scheme need not model zero edges
(\cite{Kerrer:Newman}), where as MMSB style approaches~\cite{Airoldi:2008:MMS:1390681.1442798}
 must explicitly model them.
A further set of
parallelization in inner optimization loops of local variational parameters
pushes the learning speed even more. To our knowledge, this work represents the largest 
model-based effort for analyzing social graph that takes into account both user contents and transient user thread structures. 

\vspace*{-0.5\baselineskip}
\paragraph{Contributions}
\vspace*{-0.5\baselineskip}
\begin{itemize}
  \item By directly modeling active sub-networks, this work provides new insights into how information about varying interactions across threads amongst users can leads to discovery of novel communities that were ignored by less-sophisticated methods such as spectral clustering and MMSB.
 \item By combining the text and network along with discussion structures in the active sub-networks, our method is able to outperform LDA, MMSB, Matrix Factorization and bag of words model in link prediction on three datasets we studied.
\item We developed a highly scalable inference algorithm for our proposed model, which is able to achieve convergence in
matter of hours for users and threads that are in order of a million despite the time complexity of \order{users$\times$users$\times$threads}, and we validated correctness of our inference algorithm on synthetic experiments on consistency and stability in model recovery. 
\end{itemize} 

\ericx{Here is a comment: you are using the word "model" in very inaccurate way all over the place. What you really meant can be the graphical model, the algorithm you devised to solve the inference/learning problem with the model, or just the overall method/approach we propose here. I have corrected such confusion in the intro, but please be careful in the main text.}


% \section{Graphical Model \& Generative Story}
% \section{Approach}
\section{Forum Structure Modeling}
\label{sec:approach}
Online forums have a specific discussion structure  that
provides a lot of context to all the interactions occurring among the users.
Here we describe a typical forum discussion scenario.

\subsection{Discussion in online forums}
When two users interact in a thread or through a post
they play certain conversational roles and project their specific identity.
It is valuable to know what conversational roles each plays (which topic or
community they each belong to) in that interaction. When a user $u$ is
representing community $c$ out of all the communities that he is part of and
participates in a discussion, he tailors his post content accordingly to suit
the explicit or implicit community and discussion norms. Knowing the style of
community specific text content provides a lot of information about that community 
in general. It also provides information about what role user $u$ plays when he
is in community $c$ and engaged in a specific discussion thread $t$. This helps
in better modeling of communities since it takes into account the actual 
interaction dynamics. In online forums multi-user interactions occur a lot i.e.
in a thread a user can post by addressing another specific user but he is
also addressing other users in the thread explicitly or implicitly (via either
gearing his replies to address other users' concerns into consideration or
addressing them directly in the post).
Modeling this phenomenon would bring the model closer to realities of online discussions.
This can be modelled by aggregating users posts across a thread,
though not across the whole of the forum. We will elaborate on this more in the
generative story of our model. 
% Another interesting property of such structured
% conversations is that there is an inherent bias towards the thread starter (or
% topic of the thread). Each user makes his contributions to the threaded
% discusison in light of this context. A model that takes this phenomena into
% account is closer to reality. 
% (\comment{This would be very relevant for post-and-response
% forums in our dataset such as Reddit and Stack Overflow. Right now our graphical
% model doesn't support this but it would be interesting to see how this can be brought in. 
% It might not be too difficult to do this}).
\subsection{Graphical model \& generative story}
\label{sec:gen-story}
Based on the description above our graphical model is designed as shown
in Figure~\ref{fig:finalThreadAggregationModel}. In this model
we aggregate the posts of 
a given user in a given thread $t$ into one document which has token
set $N_{t,p}$ . This helps us incorporate the knowledge that a user's post is
influenced by the posts of other users present on the thread, assuming that
he reads at least some of them.
The generative process for the model is as follows:
\begin{figure}
\centering
\includegraphics[height=4cm,width=8.5cm]{model.pdf}
\caption{\small{The proposed approach models active sub-network of users in the forum.
$U$ is the total number of user in all of the forum, $T$ is the number of
threads. $N_{T_p}$ is the total number of tokens user $p$ posted in thread $t$.
}}
\label{fig:finalThreadAggregationModel}
\end{figure}

%Assuming that there are total $N_t$ users in the thread $t$.  
% \begin{itemize}
%   \item For each Thread $t$,
\begin{itemize}
  \item For each user $p $,
  \begin{itemize}
    \item Draw a $K$ dimensional mixed membership vector 
    $\overset{\rightarrow}{\uppi}_{p} \sim$ Dirichlet($\alpha$). 
	This represents the community membership vector of the user $p$
	across $K$ communities.
  \end{itemize}
  \item for each topic pair $g$ and $h$,
    \begin{itemize}
    \item Draw $B(g,h) \sim Gamma(\kappa,\eta)$; where $\kappa, \eta$ are
	parameters of the gamma distribution. $B_{g,h}$ provides the interaction
	strength between users of community $g$ and $h$.
  \end{itemize}

  \item For each pair of users $(p, q)$ and each thread $t$,
  \begin{itemize}
    \item Draw membership indicator for the initiator, 
    $\overset{\rightarrow}{z}_{(p \rightarrow q,t)} \sim$
    Multinomial($\uppi_{p}$). 
	$\overset{\rightarrow}{z}_{(p \rightarrow q,t)}$ indicates
	the community user $p$ lies in when he initiates a conversation
	with user $q$ in thread $t$.
    \item Draw membership indicator for the receiver,
    $\overset{\rightarrow}{z}_{(p \leftarrow q,t)} \sim$
    Multinomial($\uppi_{q}$).
	$\overset{\rightarrow}{z}_{(p \leftarrow q,t)}$ indicates the
	community user $q$ lies in when he is interacting with user $p$
	in a conversation initiated by user $p$ in thread $t$.
    \item Sample the value of their interaction, $Y(p,q,t) \sim$
    Poisson( ${\overset{\rightarrow}{z}}^{\top}_{(p \rightarrow q,t)}
    B~\overset{\rightarrow}{z}_{(p \leftarrow q,t)}$ ). This is 
	the number of times user $p$ replies to user $q$ in thread $t$.
	\end{itemize}
	\item For each user $p \in t$,
	\begin{itemize}
	  \item Form the set $\delta_{t,p}$ that contains all the users that p
	  interacts to on thread $t$,
	  \begin{itemize}
	    \item For each word $w \in N_{t,p}$, 
	    \item Draw $z^{'}_{t,p,w}$ from $Dirichlet(\sum_{\forall q\in \delta_{t,p}} z_{(t,p
	    \rightarrow q)})$. This is the topic indicator for token $w$ of user's aggregated
		post in thread $t$.
	    \item Draw $w \sim \phi(w|\beta,z^{'}_{t,p,w}) $.
	  \end{itemize}
  \end{itemize}
\end{itemize}  
% \end{itemize}

The use of Poisson distribution for $Y(p,q,t) \sim$ \\
    Poisson(${\overset{\rightarrow}{z}}^{\top}_{(p
\rightarrow q,t)} B~\overset{\rightarrow}{z}_{(p \leftarrow q,t)}$) (the 
network edge between the user's $p$ and $q$) besides modelling non-binary edge
strength enables the model to ignore non-edges between users ($Y_{t,p,q}$) and
thus achieve faster convergence~\cite{Kerrer:Newman}. In MMSB style community block models, 
	 non-edges are to be modelled
explicitly.
%\begin{equation}
\small
\begin{align}
	&	\log L = \log \! P(Y, W, Z_{\leftarrow}, 
Z_{\rightarrow}, \Pi, B, \beta | \alpha, \eta, \theta, \alpha). \nonumber \\
      &= \sum_{t} \bigg[ \sum_{p,q} \! \log P(Y_{t,p, q} | Z_{t,p \rightarrow q} 
      , Z_{t,p \leftarrow q}, B) \nonumber  
      + \sum_{p,q} \log P(Z_{t, p \rightarrow q} | \Pi_q) \\ \nonumber 
      & + \sum_{p,q} \log \! P(Z_{t, p \leftarrow q} | \Pi_{q}) \bigg] 
%      + \sum_{p} \log \! P(\Pi_{p} | \alpha)  \\ \nonumber 
      + \bigg[ \sum_{t=1}^{T} \! \sum_{p \in t} \sum_{i=1}^{N_{T_{p}}} 
      \log \! P(w_{t,p,i} | Z'_{t,p,i}, \beta) \\ \nonumber 
      & + \sum_{t=1}^{T} \sum_{p \in t} \sum_{i=1^{N_{T_{p}}}} \log \! 
      P(Z'_{t,p,i} | \bar{Z}_{t, p \rightarrow q}) \bigg] + \sum_{k} \log P(\beta_{k} | 
      \eta)  
      \\ & +  \sum_{g,h} \log P(B_{g,h} | \kappa, \theta)
	  + \sum_{p} \log \! P(\Pi_{p} | \alpha)
 \label{eqn:LL}
\end{align}
%\end{equation}
\normalsize
The log-likelihood of the model described in \ref{sec:gen-story} is given
 above and derived in detail in the appendix ~\ref{eqn:LL}.

\vspace*{-1\baselineskip}
\small
\begin{align}
	&q = \prod_{p}q(\Pi_{q} | \gamma_{p}) \prod_{t} \bigg[ \prod_{p, q} \! 
q(Z_{t, p \rightarrow q}, Z_{t, p \leftarrow q} | \phi_{t,p,q})  \nonumber\\ 
\cdot &\prod_{p \in t} \prod_{i=1}^{N_{T_{p}}} q(Z'_{t,p,i} | \chi_{t,p,i})
\bigg] %\nonumber \\
\cdot \prod_{g,h} q(B_{g,h} | \nu_{g,h} \lambda_{g,h}) \prod_{k} q(\beta{k} |
\tau_{k}).
\label{eqn:variationalQ}
\end{align}
\normalsize
We use variational approximation to maximize log-likelihood.
Equation~\ref{eqn:variationalQ} above is the approximation of the log-likelihood and we
use structured mean field~\cite{Xing_et_al:2003} to maximize parameters of $q$.
The local variational parameters, $\phi$ (MMSB parameters) and $\chi$ (LDA)
parameters, are maximized using equations \ref{eqn:phiUp} and \ref{eqn:chiUp}.
$\Delta_{\phi^{'}_{t,p,g,h}}$, $\Delta_{\chi^{'}_{t,p,g,h}}$ 
%are defined by equations ~\ref{eqn:phiDelta} and~\ref{eqn:chiDelta} respectively.
\vspace*{-1\baselineskip}
\small
\begin{align}
	&\phi_{t,p,g,h} \propto e^{\Delta_{\phi^{'}_{t,p,g,h}}}.
\label{eqn:phiUp}                                 \\
&\chi_{t,p,i,k} \propto e^{\Delta_{\chi^{'}_{t,p,g,h}}}.
\label{eqn:chiUp}
\end{align}
\vspace*{-1\baselineskip}
\begin{align}
\Delta_{\phi^{'}_{t,p,g,h}} &= y_{t,p,q}( \log \! \lambda_{g,h} + 
\Psi(\nu_{g,h})) - \nu_{g,h} \lambda_{g,h} - \log \! (y_{t,p,q}!)
\nonumber \\ & + \Psi(\gamma_{p,g}) - \Psi(\sum_{g} \gamma_{p,g})
\nonumber \\  & + \Psi(\gamma_{q,h}) - \Psi(\sum_{h} \gamma_{q,h})
\nonumber \\  & + \omega\sum_{i=1}^{N_{T_{P}}} \! \chi_{t,p,i,g} 
\bigg[ \ln \! \frac{\epsilon}{\delta_{p,t}} - \frac{1}{\delta_{t,p}} 
+ \ln \bigg( 1 + \frac{\epsilon}{\delta_{p,t}} \bigg) 
\cdot \frac{1}{\delta_{t,p}} \bigg].
\label{eqn:phiDelta}
\end{align}
\vspace*{-1\baselineskip}
\begin{align}
\Delta_{\chi^{'}_{t,p,g,h}} &= \bigg[ \Psi(\tau_{k,w_{t,p,i}}) - 
\Psi(\sum_{w_{t,p,i}} \! \tau_{k, w_{t,p,i}}) \bigg]
\nonumber \\  & + \ln \! (\frac{\epsilon}{\delta{p,t}}) \frac{1 - 
\sum_{q,h} \! \phi_{t,p,q,k,j}}{\delta_{t,p}} 
\nonumber\\  & + \frac{\sum_{q,h} \phi_{t,p,q,k,h}}{\delta_{t,p}} 
\ln \! (1 + \frac{\epsilon}{\delta_{t,p}}).
%\nonumber\\  & - 1 - \log \! \chi_{t,p,i,k}
\label{eqn:chiDelta}
\end{align}
\normalsize
and the traditional variational updates for global parameters $\gamma, \nu, \lambda$
(MMSB) and $\tau$ (LDA) are defined in apendix.
%using equations
%\ref{eqn:gammaUp}, \ref{eqn:nuUp}, \ref{eqn:lambdaUp} and \ref{eqn:tauUp}
%respectively (details are in the appendix).
% ~\ref{sec:appendix}.
\vspace*{-0.5\baselineskip}
\subsection{Terminology} 
\label{sec:term}
There is a difference to be made between
community-topic, word topic and user roles. Community topic is the 
$\pi$ vector that we get from
the model (figure~\ref{fig:finalThreadAggregationModel}) that decides the 
membership proportion of a user in
different latent communities. Word topic is the $\beta$ vector of word topic
proportions from the LDA component of the model,
figure~\ref{fig:finalThreadAggregationModel}. There is a one to one
correspondence between $\beta$ and $\pi$ vectors as seen in
figure~\ref{fig:finalThreadAggregationModel}. $\beta$ helps us in identifying
what contents users are generally interested in a given latent community.
 A user conversational role (or simply role) is a specific 
 configuration of $\pi$. It can be just the case
that a role '$r$' might be the $\pi$ vector where $r$-th coordinate is 1
and all else are 0 out of the total K coordinates, i.e. it predominantly relates
to that $r$-th latent community.


\section{Scalable Estimation}
\label{estimation}

\begin{figure}
\begin{center}
\includegraphics[height=6cm,width=9cm]{SpeedOptimization.pdf}
\end{center}
\caption{\small{The log-likelihood vs incremental speed optimization routine. 
The right hand plot is a zoomed in version of the left. PSSV
(Parallel Sub-sampled Stochastic Variational), SSV(Sub-sampled Stochastic
Variational), SV(Stochastic Variational) and V(Variational). Each addition of
optimization increases the speed by several orders of magnitude. The final PSSV
is 4 times faster than (V)ariational and achieves better log-likelihood too.
\abhi{explain the left and right plot relation in detail}}}
\label{fig:SpeedOptimization}
\end{figure}
The global update equations in the previous section are computationally
expensive, as we need to sum over all the updated local
variables. $U$ users with $T$ threads and vocabulary size $V$ leads to
$O(U^2T+UVT)$ local variables. Traditional sampling or 
variational estimation techniques would be prohibitively slow for such a model. In order to obtain faster convergence we
make use of stochastic variational approximation along with sub-sampling and parallelization. 
% Algorithm~\ref{algo:stochasticAlgo} describes in detail the sub-sampled SVI
% based approximate estimation of the model. 

The updates in case of SVI with sub-sampling follow a two step procedure. Step
one computes a local update for the global variables based on the sub-sampled
updated local variables. The local updates ($\gamma^{'},
\nu^{'}, \lambda^{'}$ and $\tau^{'}$) for the global variables ($\gamma,
\nu, \lambda$ and $\tau$) are

{\small
\begin{align}
\gamma_{p,k}^{'} &= \alpha_{k} + \frac{NT}{2|S_p|}\sum_{q \in S_p} \sum_{h}
\! \phi_{t,p,q,k,h} + \frac{NT}{2|S_p|}\sum_{q \in S_p} \! \sum_{g} \!
\phi_{t,q,p,g,k}. 
\label{eqn:gammaUpStoc}
\end{align}
\vspace*{-1.5\baselineskip}
\begin{align}
\nu_{g,h}^{'} &= \nu_{g,h}^{t}+\rho_\nu \frac{NT}{2|S_p|}\sum_{q \in
S_p}\frac{dL}{\partial\nu_{g,h}}.
\label{eqn:nuUpStoc}
\end{align}
\vspace*{-1.5\baselineskip}
\begin{align}
\lambda_{g,h}^{'} &= \frac{\bigg( \sum_{t} \! \sum_{p,q} \! \phi_{t,p,q,g,h}
y_{t,p,q} + \kappa_{g,h} \bigg) }{
 \bigg( \bigg( \sum_{t} \! \sum_{p,q} \! \phi_{t,p,q,g,h} \bigg) + 
\frac{1}{\theta_{g,h}} \bigg) \nu_{g,h}}.
\label{eqn:lambdaUpStoc}
\end{align}
\vspace*{-1.5\baselineskip}
\begin{align}
\tau_{p,v}^{'} = \nu_{v} + \frac{NT}{2|S_p|} 
\bigg(\sum_{w_{t,p,i}=v}^{N_{t,p}} \chi_{t,p,i,k} \bigg),
\label{eqn:tauUpStoc}
\end{align} 
}
where $S_p$ is a set of neighborhood edges of user $p$, and $N$ and $T$ are
total number of edges and threads respectively in the network. The set $S_p$ is
chosen amongst the neighbors of $p$ by sampling equal no. zero and non-zero
edges. 

\normalsize
\IncMargin{1em}

\begin{algorithm}[t]
\small
\SetAlgoLined
Input : $Y,W,c,\alpha,\theta,\kappa,\eta, S, U$\\
Initialize : $\gamma\leftarrow \gamma_0$,
$\tau\leftarrow \tau_0, \nu\leftarrow \nu_0, \lambda\leftarrow \lambda_0$\\
Equally divide all the threads in the forum into smaller sets $T_c$\\ 
\While{not converged}{
\For{$c$ processors \textbf{in parallel}}{
	pick $C_t$: a random subset of threads in $T_c$ of size $S$\\
	\For{each $t\in C_t$}{ 	
		pick $U_t$: a random subset of users in $t$ of size $U$ \\
		\For{each $p\in U_t$}{ 	
			$\forall q\in~neighborhood~\delta_{t,p}$\\
			\While{$\phi~\&~\chi$ not converged}{
				get new $\phi_{t,p\rightarrow q}$, $\phi_{t,p\leftarrow q}$,
				$\phi_{t,q\rightarrow p}$, $\phi_{t,q\leftarrow p}$\\
				and $\chi_{t,p,i} \forall i \in N_{t,p}$ \\
				iterate between $\phi$ and $\chi$ using
				equations~\ref{eqn:phiUp}~and~\ref{eqn:chiUp}			
			}
		}
	}
}
aggregate $\phi$ and $\chi$ obtained from different processors.\\
get local update $\gamma^{'}$,$\tau^{'}$, $\nu^{'}$, $\lambda^{'}$ via
stochastic approximation of traditional variational updates
(defined in appendix)\\
%equations~\ref{eqn:gammaUp},\ref{eqn:tauUp},\ref{eqn:nuUp},\ref{eqn:lambdaUp}.\\
get global updates of $\gamma$,$\tau$, $\nu$, $\lambda$; e.g. $\gamma^{t+1} =
(1-step)\gamma^{t}+(step)\gamma^{'}$ \\
Similarly globally update $\tau,\nu,\lambda$ as above using
equation~\ref{eqn:globalUpStoc}.
}
\label{algo:stochasticAlgo}
\caption{{\small PSSV: Parallel Sub-sampling based Stochastic Variational inference for
the proposed model. Arguments $Y$ and $c$ are the input data and number of processors
respectively where as rest of the 
arguments are parameters that are tuned over a heldout set.}}
\end{algorithm}



In step two of the sub-sampled SVI the final update of global variable is
computed by the weighted average of the local updates of the global variable and
the variables value in the previous iteration:
%{\small
\begin{align}
\mu^{t+1} = (1-\xi^t)\mu^{t} + \xi^t\mu^{'}. 
\label{eqn:globalUpStoc}
\end{align} 
\normalsize
%\vspace*{-1\baselineskip}
where $\mu$ represents any global variable from $\lambda, \nu, \gamma, \tau$.
 $\xi^t$ is chosen appropriately using SGD literature and is
decreasing.  $\xi^t$ is standard stochastic
gradient descent rate at iteration $t$, also expressed as $\xi^t =
\frac{1}{(t+\zeta)^{\rho}}$~\cite{conf/nips/GopalanMGFB12}. $\zeta$ and
$\rho$ are set as 1024 and 0.5 respectively for all our experiments in the
paper, and $t$ is the iteration number. 

We achieve further speed by parallelizing the text ($\chi$) and network
($\phi$) local variational updates. This is achievable  as the
dependency between $phi$ and $\chi$ parameters (defined in
equations~\ref{eqn:chiDelta} and \ref{eqn:phiDelta}) allows us to parallelize their
variational updates.
Algorithm~\ref{algo:stochasticAlgo} describes the parallelized SVI with
sub-sampling updates for the local parameters. This algorithm is 
based on the theoretical insights from~\cite{Hoffman:2013:SVI} and is 
quite stable as we will see in section~\ref{sec:results}. 
Figure~\ref{fig:SpeedOptimization} 
shows a plot of how the final (p)arallel (s)ub-sampling based (s)tochastic
(v)ariational (PSSV) inference is faster than each of its individual components.
SO dataset described in section~\ref{sec:dataset} is used as the data for this 
experiment.
The number of parallel cores used in the PSSV scheme is four whereas its one
for the rest of the three.
The amount of sub-sampled forum threads is 400 and the total number of threads is 14,416.
All the schemes in the graph start with the same initialization values of the
hyper-parameters. PSSV is at least twice as fast as the nearest scheme besides
obtaining the best log-likelihood of all the four at the point of convergence.
The SV (stochastic variational) samples one thread at a time and therefore
takes some time in the beginning to start minimizing the objective value
(negative log likelihood).
The objective value increases in the first few iterations for SV. The number of
iterations to be done by SV is very large but each iterations takes the smallest time of
all four. The V (variational) scheme takes the least number of iterations to
converge though its iterations are the most time consuming as it has to go
through all the 14,416 threads in every iteration. 


\begin{table}
\begin{center}
\begin{tabular}{c|c|c|c|c|}
 & users & threads & posts & edges\\\hline
 TS & 22,095 & 14,416 & 1,109,125 & 287,808\\\hline
 UM & 15,111 & 12,440 & 381,199 & 177,336\\\hline
 SO & 1,135,996 & 4,552,367 & 9,230,127 & 9,185,650\\\hline
\end{tabular}
\label{tab:dataStats}
\end{center}
\caption{Dataset statistics. SO mostly has edges with weight one.}
\end{table}




\section{Datasets}
\label{sec:dataset}

\begin{figure}
\begin{center}
%\includegraphics[height=6cm,width=9cm]{EdgeDistribution.pdf}
\includegraphics[height=6cm,width=9cm]{figs/datastats.pdf}
\end{center}
\caption{\small{Distribution of different edge weights over the 3 datasets. SO 
predominantly consists of edges with weight one. Right hand plot is a scaled
version of the left. The label '11' contains edge weights of 11 and above.}}
\label{fig:EdgeDistribution}
\end{figure}
We analyse three real world datasets corresponding to two different forums: 1)
Cancer-ThreadStarter, 2) Cancer-UserName, and 3) Stack Overflow. To test the
stability of the model we use a synthetically generated dataset. 
The Cancer forum \footnote{\url{http://community.breastcancer.org}} 
is a self-help community where users who either have cancer, are 
concerned they may have cancer, or care for others who have cancer, 
come to discuss their concerns and get advice and support.
StackOverflow is an
online forum for question answering primarily related to computer science. We
use the latest dump of Stack
Overflow~\footnote{\url{http://www.clearbits.net/torrents/2141-jun-2013}}. In
each of these datasets a user posts multiple times in a thread and all these
posts are aggregated into one bigger posts per thread as defined in
section~\ref{sec:gen-story}.
Number of times a user $u$ replies to user $v$ in thread $T$ is the edge weight
of edge $u\rightarrow v$ in thread $T$.
Table~\ref{tab:dataStats} gives the distributions of edges, posts, users and
threads in the three datasets used. 
\vspace*{-0.5\baselineskip}
\paragraph{Cancer-ThreadStarter (TS)} In the Cancer forum, the conversations happen
in a structured way where users post their responses on a thread by thread
basis. Every thread has a thread starter that posts the first message and
starts the conversation. 
We construct a graph from each thread by drawing a link from each participant on 
the thread to the participant who started the thread
This graph has 22,095 users and 14,416 Threads. 
% \comment{Put number of
% posts and edges}
\vspace*{-0.5\baselineskip}
\paragraph{Cancer-Username Mention (UM)} Users call each other by 
their usernames (or handle assigned to them in the
forum) while posting in many cases. We create a graph where in an edge between
user $u$ and user $v$ in thread $t$ means that user $u$ calls user $v$ by
username in thread $t$. This graph has 15,111 users and
12,440 threads.
% \comment{Put number of posts and edges}
\vspace*{-0.5\baselineskip}
\paragraph{Stack Overflow (SO)} In Stack Overflow users ask 
questions and then other users reply with their
answers. We obtain the ThreadStarter graph from this structure. This dataset has
have 1,135,996 users and 4,552,367 threads.
% \comment{Put number of posts and edges}
\vspace*{-0.5\baselineskip}
\paragraph{Synthetic data} We generate a synthetic dataset 
using the generative process defined in
section~\ref{sec:gen-story}. For stability experiments 
we fix users at 1000 and threads at 100. The number of 
posts and edges vary depending on the choice of priors $\alpha$ and $\eta$.
For scalability experiments we vary the number of user and threads.  
% \comment{Put number of posts and edges as well as the parameters}

\begin{figure}
\begin{center}
\includegraphics[height=6cm,width=9cm]{3LLPlots.pdf}
\end{center}
\caption{The log-likelihood over heldout set for the fully tuned model on the
3 datsets}
\label{fig:finalLLheld}
\end{figure}

\begin{table}
\begin{center}
\begin{tabular}{c|c|c|c|c|c|c|}
 & $\alpha$ & $\omega$ & $\theta$ & $\kappa$ & $\eta$ & $K$\\\hline
 TS & 0.05 & 1e-4 & 2.5$\sim$1.5 & 2.5$\sim$1.5& 0.05 & 10\\\hline
 UM & 0.05 & 1e-3 & 2.0$\sim$1.0 & 2.0$\sim$1.0 & 0.05 & 10\\\hline
 SO & 0.05 & 1e-2 & 1.0$\sim$0.5 & 1.0$\sim$0.5 & 0.05 & 20\\\hline
\end{tabular}
\label{tab:tunedParameters}
\end{center}
\caption{Tuned values for the parameters. $\theta$ and $\kappa$ are matrices
and $a\sim b$ assigned to them means diagonal values are $a$ and non-diagonals
are $b$. $K$ is the number of topics}
\end{table}

\section{Experiments}

\subsection{Experimental Setup and Evaluation}
\label{sec:setup}
We divide each dataset into three subsets: 1) the training set, 2) the heldout set,
and 3) the test set. We learn our model on training set and tune our priors 
($\alpha, \eta, \kappa, \theta$ etc.) on heldout set. The split is done over the
edges where 80\% of the edges are in training and rest 20\% are divided amongst
heldout and test equally. By splitting we mean that in the training set we
omit the 20\% of the edges from the graph and jeep only rest 80\% and similary 
in the heldout and test set. 
%For the two cancer datasets we only predict non-zero edge weights whereas for
%the Stack Overflow we predict zero as well as non-zero 
%edge weights.
Graph \ref{fig:EdgeDistribution} shows the distribution of edge weights in
cancer and Stack Overflow dataset. We chose Stack Overflow to predict
zero weights since it has large number of edges with very low weights,
predominantly weight one. Predicting zero as well as non-zero edge weights
demonstrates that the model is versatile and can predict a wide range of 
edge-weights. In addition to 20\% of the total
non-zero edges we randomly sample equal number of zero edges from the graph for
the SO held and test set.
The optimization objective for learning is defined in
equation~\ref{eqn:VarLowerBound}.

A link prediction task is incorporated to demonstrate the model's effectiveness.
It is a standard task in the area of graph clustering and social networks in
particular. Researchers have
used it in the past to demonstrate their model's learning 
ability~\cite{Nallapati:2008:JLT:1401890.1401957}.   
The link prediction task works as an important validation of our model. If the
proposed model performs better than its individual parts then it can be safely
concluded that it extracts important patterns from each of its building
blocks. Moreover it adds validity to the qualitative analysis of the results. 

\vspace*{-0.5\baselineskip}
\paragraph{Link-prediction} 
  
 We predict the edge-weight of the edges
present in the test set. The predicted edge, $\hat{Y}_{t,u,v}$, between users $u$ and
$v$ in thread $t$ is  defined as 
\vspace*{-1\baselineskip}
\small
\begin{align}
\hat{Y}_{t,u,v} = \pi^T_uB\pi_v\label{eqn:prediction}.\\
B=\nu.*\lambda\label{eqn:blockMat}
\end{align}
\normalsize
and the prediction error is the $rmse$, defined as given predicted edge
$\hat{Y}_{t,u,v}$ and the actual edge $Y_{t,u,v}$.
\vspace*{-0.5\baselineskip}
\small
\begin{equation}
	rmse=\sqrt{\frac{\sum_{(u,v)\in E}(\hat{Y}_{t,u,v}-Y_{t,u,v})^2}{|E|}}.
\end{equation}
\normalsize

\begin{figure}
\begin{center}
\includegraphics[height=6cm,width=9cm]{TopicVariationsLocal.pdf}
\end{center}
\caption{\small{Number of local variations in topic proportion on a per user per thread
level. The axis is percentage variation (from 10 to 90 percent). The right hand
plot is a scaled in version of the left.}}
\label{fig:localTopicVariations}
\end{figure}

\begin{center}
\begin{figure*}
\includegraphics[height=7cm,width=18cm]{SimilarityMatTSAnnotated.png}
\caption{Adjacency matrix of users sorted by clusters. Left side is 
clustered by MMSB and right side is clustered by our model using user's dominant
role as cluster index over TS dataset. Our model is able to correctly cluster 3K
additional users that MMSB doesnt assign any dominant cluster (or role) and
discovers a new role (Cluster-6).}
\label{fig:similarityMatTS}
\end{figure*}
\end{center}
The summation is over the edge set $E$ in the test (or heldout) set. The block matrix
$B$ described in equation~\ref{eqn:blockMat} is well defined for both MMSB and
the proposed model. Hence the prediction is obtained for the active network modelling without
LDA (just MMSB component) and with LDA. We created an artificial
weighted Identity matrix for LDA $\hat{B}=m*I$. It is a diagonal matrix with
all element values $m$. For every user $u$ and every thread $t$ the topics discovered over the
posts of $u$ in $t$ is used as the vector $\pi_u$ in
equation~\ref{eqn:prediction} for prediction. A diagonal $B$ is desirable in
block models as it provides clean separation among the clusters
obtained~\cite{Airoldi:2008:MMS:1390681.1442798}.
The value of $m$ is tuned over heldout set. 
The MF model is the normal matrix factorization model. We obtain the 
matrix factors on training set and use them to predict the edge weight between
users on the test and heldout sets. It doesn't use text part of the data.
The BoW model is a simple bag of words model that takes each user $u$'s
aggregated post across the forum and constructs a vector $V_u$ of words. The 
edge weight $w_{u,v}$ predicted between users $u$ and $v$ is $w_{u,v}=cosine\_sim(V_u,V_v)*b$
where $cosine\_sim(x,y)$ is cosine similarity between vectors $x$ and $y$ and $b$ is tuned
over heldou set. We define a basic baseline that
always predicts the average weight ($\bar{Y}$) all the edges, zero (Stack
Overflow) or non-zero (Cancer), in the heldout or test set.
\vspace*{-0.5\baselineskip}
\small
\begin{equation}
	rmse_{baseline}=\sqrt{\frac{\sum_{(u,v)\in E}(\bar{Y}_{t,u,v}-Y_{t,u,v})^2}{|E|}}.
\end{equation}
\normalsize

\vspace*{-0.5\baselineskip}
\paragraph{Community Discovery}
Besides providing qualitative insights into the communities dicovered
we also provide a quantitative evaluation of the communities. We use 
$\rho$, a variation of Davies-Bouldin index~\cite{Davies:1979} 
\vspace*{-0.5\baselineskip}
\small
\begin{align}
	\rho=\frac{1}{K} \sum_{c=1}^{c=K}\frac{d_{c,in}}{d_{c,out}}
\end{align}
\normalsize
where $d_{d,in}=\frac{\sum_{u,v\in c}Y_{u,v}}{|E_{in}|}$ is the sum of the 
edges of all user pairs inside the community $c$ and $|E_{in}|$ is the number of 
all such edges. $d_{c,out}=\frac{\sum_{u\notin c,v\in c}Y_{u,v}}{|E_{out}|}$ is the 
sum of the edges of user pairs where one of the users lies in $c$ and the other 
outside $c$ and $|E_{out}|$ is the number of such edges. 
$\rho$ is the average ratio of the community densities and measures 
the compactness of the community.
We compare our model with simple MMSB and Graclus, a scalable spectral clustering
algorithm~\cite{Dhillon:2005}

\vspace*{-0.5\baselineskip}
\paragraph{Parameter tuning}
We tune our parameters $\eta, \kappa,
\theta, \alpha$, and $K$ (number of community-topics) over the held set.
$\omega$, the parameter to balance the contribution of the text side to the 
network side is tuned over the
heldout set. It is used in the local variational update of
$\phi$(equation~\ref{eqn:phiDelta}).
Equation~\ref{eqn:phiDelta} contains a summation term over all the tokens
$\sum_{i=1}^{N_{T_p}}\chi_i$ in the per user per thread document and if  not
balanced by $\omega$ will dominate the rest of the terms. The constant $\epsilon$ used in
equation~\ref{eqn:phiDelta} and~\ref{eqn:chiDelta} is a smoothing constant and
is fixed at a low value. The six quantities, $\alpha, \omega, \theta, kappa,
\eta$ and $K$ are tuned in that sequence. $\alpha$ is tuned first keeping rest
constant then $\omega$ and so on where each next to be tuned parameter uses
values of already tuned parameters. Table~\ref{tab:tunedParameters}
shows the final values of all the parameters.Figure~\ref{fig:finalLLheld} 
shows plot of tuned log-likelihood over the three
datasets against time. UM being the smallest of the two takes
the least amount of time. Tuning for the parameters of MF, BoW, Graclus and LDA
is also done similarly on the heldout set.

\paragraph{System details} The machine used in all the experiments in this paper
is ``Intel(R) Xeon(R) CPU E5-2450 0 @ 2.10GHz'' 16 corewith 8GBs of
RAM per core. The operating system is Linux 2.6.32 x86\_64. 

\subsection{Results}
\label{sec:results}

\begin{table}
\begin{center} 
\begin{tabular}{p{1cm}|p{0.7cm}|p{0.7cm}|p{0.7cm}|p{0.7cm}|p{0.7cm}|p{0.7cm}|}
  & UM held & UM test & TS held & TS test & SO held & SO test \\\hline
Our Model & \textbf{1.213} &\textbf{1.194} &\textbf{2.562} & \textbf{2.548} & 
\textbf{0.306}& \textbf{0.311} \\\hline 
MMSB &1.450& 1.502 & 2.984& 2.881 &0.421 & 0.434 \\\hline 
LDA &1.793& 1.731	&3.725 & 3.762 &0.466 & 0.479\\\hline
MF & 1.412 & 1.468 & 2.925 & 2.905 & 0.413 & 0.420 \\\hline
BoW &1.834 & 1.851 & 4.012 & 4.105 & 0.497 & 0.491 \\\hline
Baseline &1.886& 1.983 &4.504 &4.417&0.502& 0.509\\\hline
\end{tabular}
\caption{\small{Link prediction results over the three datsets}}
\label{tab:predictionResults}
\end{center}
\end{table}

\begin{table}
\begin{center} 
\begin{tabular}{p{1cm}|p{0.7cm}|p{0.7cm}|p{0.7cm}|p{0.7cm}|p{0.7cm}|p{0.7cm}|}
  & UM held & UM test & TS held & TS test & SO held & SO test \\\hline
Our Model & \textbf{1.963} &\textbf{1.91} &\textbf{1.932} & \textbf{1.935} & 
\textbf{1.898}& \textbf{1.915} \\\hline 
MMSB &1.421& 1.417 & 1.443& 1.416 &1.387 & 1.404 \\\hline 
Graclus &1.598& 1.531	&1.460 & 1.457 &1.461 & 1.438\\\hline
\end{tabular}
\caption{\small{Average density ratios $\rho$ of different community 
discovery models over three datasets}}
\label{tab:rhoResults}
\end{center}
\end{table}

\begin{figure}
\centering
  \includegraphics[width=0.7\linewidth]{synth.pdf}
\caption{\small{Time taken in hours with increasing number of threads and users}}
\label{fig:syntheticScalability}
\end{figure}
 
\vspace*{-0.5\baselineskip}
\paragraph{Link prediction} Table~\ref{tab:predictionResults} shows the link
prediction results on heldout and test set for six prediction
models. The proposed approach to model thread 
level conversational roles outperforms
all of the other models. LDA performs poorer than MMSB since LDA does not 
explicitly model network information. MF perfroms slightly better than MMSB 
where as BoW perfoms worse than LDA and slightly better than the baseline. 
This means that text topic similarity contains valuable information regarding 
the users.

\vspace*{-0.5\baselineskip}
\paragraph{Cancer dataset}
Figure~\ref{fig:localTopicVariations} shows the number of times the global role
of a user is different from the thread level role that he plays. It is 
interesting to see that the variation between global and thread level role
assignment is high among all the datasets. A model that ignores this local vs
global dynamics tends to lose a lot of information.
Figure~\ref{fig:similarityMatTS} shows the plot of the user by user adjacency 
matrix for TS dataset. The users are sorted based on the community-topic cluster
(roles) assigned by the respective models (our model and MMSB model). The number of
community-topics are 10 and every user is assigned the dominant
community-topic, $\pi$, (role) that they have more than 50\% of chance of lying in. A user 
is discarded if he
doesn't have the said dominant role. Thus here user role and community are the same
as explaine in section~\ref{sec:term}.
Our model is unable to assign a clear 
role to 3.3K users and
the MMSB approximately to 6.3K users out of 22K. Based on the topics assigned,
users are sorted and their adjacency matrix shows clean clustering along the
block diagonals.
As seen in the figure, the combined model is able to effectively find the
primary roles (dominant topic) for the extra 3K users that the MMSB model was
unable to provide for. Besides a new role (Role 6) that is not accounted for by
MMSB or Graclus is discovered by the proposed model (figure~\ref{fig:similarityMatTS}).
Users that predominantly have role 6 on a global scale tend to vary their roles
often on a thread level, i.e. their topic probabilities change quite often. The
average change in topic probabilities per role per user-thread pair across the
10 roles discovered in TS is 30.6\%; for role 6 it is 41.5\% (highest of all the roles). 
This means
that this role is very dynamic and an active sub-network modelling helps here as 
it captures the
dynamism of this entity. From figure~\ref{fig:similarityMatTS} cluster of roles
4, 5, 6, 7 and 8 are the largest. 
Graclus discovers a similar set of communities as MMSB model. It also misses
the community 6. Table~\ref{tab:rhoResults} shows that Graclus communities 
are marginally more compact than MMSB but our model has the most compact 
communities. This is due to the fact that besides discovering role 6 and putting all 
such users in a separate role it also takes into account local variations
in the user roles that helps it find better boundaries for other communities.
%\abhi{I will also put here the coommunities discovered by spectral clustering
%and make a contrast. To contrats with the better communties discovered by our model
%I will use the within-cluster-weight/without-cluster-weight and show that our
%model is better.}
%Table~\ref{tab:top15WordsTS} corresponds to
%top 15 words corresponding to these roles. 
A detailed inspection reveals that role 4, role 7 and role 8 are
related to discussion regarding cancer where as role 5 is related to
conversations regarding spiritual and family matters.  But role 6 does not seem
to be related to any specific type of conversation. It is free flowing and has
lots of non specific discussion threads where users play games,
discuss cancer stages, talk about religious beliefs, share stories etc. 
This tells us that there is a cornucopia of
discussions happening in this role with no specific matter at hand. 
Users who are
predominantly in this role tend to post across many discussion threads and
variety of conversation topics. This role is only detected by our model due to the
fact that it takes into account the dynamics of such a role.

\begin{figure*}
\centering
  \includegraphics[width=1\linewidth]{piSOAnnotated.png}
\caption{\small{The 20 roles assigned to users in the stack overflow
dataset. The numbers at the vertex are the role numbers. Due to the large
number of roles we visualize them 5 at a time with first 5 first then second
5 and so on.
We can see that the roles are separated cleanly and clustered around the
pentagon corners.}}
\label{fig:SOclusters}
\end{figure*}

\begin{figure}
\centering
  \includegraphics[width=0.7\linewidth]{synth.pdf}
\caption{\small{RMSE vs $\alpha$ and $\eta$ for the synthetic dataset. The X-axis is
$\alpha$ or $\eta$ values and the Y-axis is the RMSE of the recovered $\pi$}}
\label{fig:syntheticRMSE}
\end{figure}
 
 
%\begin{table}
%\begin{center} 
%\begin{tabular}{c|c|c|c|c}
%Topic 4  & Topic 5 & Topic 6 & Topic 7 & Topic 8 \\\hline
%side            &same   &their  &surgeon        &radiat \\\hline
%test            &life   &live   &everi  &anoth\\\hline
%took            &tell   &happi  &found  &doctor\\\hline
%away            &mani   &mayb   &down   &problem\\\hline
%left            &famili &sorri  &alway  &pleas\\\hline
%support         &prayer &best   &while  &person\\\hline
%doesn           &though &check  &home   &kind\\\hline
%seem            &ladi   &news   &bodi   &soon\\\hline
%move            &until  &question       &these  &each\\\hline
%almost          &wish   &dure   &bone   &hard\\\hline
%scan            &someon &deal   &mean   &might\\\hline
%idea            &under  &case   &came   &medic\\\hline
%studi           &felt   &mind   &posit  &herceptin\\\hline
%guess           &where  &seem   &drug   &share\\\hline
%diseas          &nurs   &haven  &send   &free\\\hline
%\end{tabular}
%\caption{top 15 words for topics corresponding to top 5 biggest role in
%TS.}
%\label{tab:top15WordsTS}
%\end{center}
%\end{table}

%   \comment{dig in
% detail as to what this cluster is and show it in the context of local vs global roles}

%  \comment{Put interpretations of the clusters, UserName and
% ThreadStarter}\\
% \comment{Put the top 10 words in each topics}\\
% \comment{Put Block matrix to demonstrate better resolution of cluster}\\
% \comment{Put nice visualization for some per user-thread topic proportions.}\\

\vspace*{-0.5\baselineskip}
\paragraph{Stack Overflow} 
The optimal topic number for SO dataset is 20 community-topics as noted in
table~\ref{tab:tunedParameters} and the number of users are 1.13 million. It is
difficult to visualize the user-user adjacency matrix of this size. The 20
topic set is divided into four sets with size 5 each. Topics 1 to 5
form set one, topics 6 to 10 form set two and so on. Every user's role is
visualized by projecting user's $\pi$ over a pentagon as shown in
figure~\ref{fig:SOclusters}. The projection uses both position and
color to show values of community-topic $\pi$ for each user.
Every user $u$ is displayed as a circle (vertex) $v_u$ in the figure where the
size of the circle is the node degree of $v_u$ and position of $v_u$ is equal to
a convex combination of the five pentagon corner coordinates $(x, y)$ that are weighted by the
elements of $\pi_u$. Hence circles $v_u$ at the pentagon's corners
represent $\pi$'s that have a dominating community  in the 5
community-topics chosen, while circles on the lines connecting the corners 
represent $\pi$'s with
mixed-membership in at least 2 communities (as only a partial $\pi$ vector is
used in each sub-graph).
All other circles represent $\pi$'s with mixed-membership in $\geq 3$
communities.
Each circle $v_u$'s color is also a $\pi$-weighted convex
combination of the RGB values of 5 colors: blue, green, red, cyan and purple. This
color coding helps distinguish between vertices with 2 versus 3 or more
communities. We observe a big black circle at the back ground of every plot.
This circle represents the user with id 22656
\footnote{\url{http://stackoverflow.com/users/22656/jon-skeet}} that 
has the highest node-degree of 25,220 in the SO dataset. 
This user has the highest all time reputation on
stack overflow and tends to take part in myriads of question answering threads.
Hence he is rightly picked up by the model to be in the middle of all the roles. 

Figure~\ref{fig:SOclusters} has a cleanly separated communities
where the nodes are clustered
around the pentagon vertices. This indicates that the model is able to find
primary roles for most of the users in the forum.  
Table~\ref{fig:localTopicVariations} shows that there is significant amount of
variation with respect to the global role of a user at a thread level. 
Modeling this variation helps our model in
getting better clusters as compared to simple MMSB and Graclus; 
this fact is apparent from
the $\rho$s obtained, table~\ref{tab:rhoResults}. 
While Graclus's $rho$ is margianlly better than MMSB, our model outperforms
it with a significant margin. 
In terms of new communities discovered, we again observe a similar pattern
as in the case of cancer dataset wherein roles that have too much diversity 
are overlooked by MMSB and Graclus. 
While roles 1, 2, 3 are related to discussions regarding database 
conversations, general coding, android
and J2EE respectively, role 4 is diverse and related to 
online blogs, Java, C++ multi threading, Unix programming and browsing apis. 
Role 5 and role 6 are related to server/client side browser based 
coding and general coding respectively. Role 7 like role 6 is diverse and 
conversations here are related to JSON, python scripts, functional programing,
Clojure. MMSB and Graclus both overlook role 6 and 7 whereas our model
discovers them as a separate community.

%\abhi{I will put here the communties discovered by simple MMSB 
%and Spectral Clutering and will make a point that our model discover 
%all the communties discoevred by these plus more as is the case for Cancer
%dataset. To contrats with the better communties discovered by our model
%I will use the within-cluster-weight/without-cluster-weight and show that our
%model is better.}


%\begin{table}
%\begin{center} 
%% \resizebox{\linewidth}{!}{
%\begin{tabular}{c|c|c|c|c|c}
%Topic 1  & Topic 2 & Topic 3 & Topic 4 & Topic 5 & Topic 6 \\\hline
%public  &code   &function       &that   &name   &array \\\hline
%valu    &should &also   &view   &method &more \\\hline
%chang   &your   &then   &thread &time   &what \\\hline
%when    &user   &object &into   &document       &system \\\hline
%event   &properti       &creat  &control        &line   &about \\\hline
%would   &blockquot      &follow &current        &element        &sure \\\hline
%list    &just   &form   &link   &differ &post \\\hline
%databas &field  &implement      &oper   &would  &each \\\hline
%there   &class  &valu   &question       &defin  &imag \\\hline
%issu    &overrid        &server &length &file   &class \\\hline
%path    &main   &veri   &creat  &result &paramet \\\hline
%display &result &string &each   &where  &applic \\\hline
%like    &size   &start  &result &more   &just \\\hline
%order   &import &java   &blog   &know   &local \\\hline
%save    &project        &android        &featur &browser        &specif \\\hline
%\end{tabular}%}
%\caption{top 15 words for word topics corresponding to first 6 role clusters in
%SO.}
%\label{tab:top15WordsSO}
%\end{center}
%\end{table}

%\abhi{Some text topic interpretation can be there but it could be removed
%in case of space crunch.}
Our model is not only better at discovering diverse roles but also roles 
that predominantly belong to one category but are obscured at times. 
If the user deviates from his dominant role
for a short while then Graclus and MMSB have difficulty assigning
a predominant role. 
Take the user 20860 in SO dataset. 20860 is globally assigned 
role 1 as the dominant role but he also takes part in coding related 
discussions sometimes. For example, 
\small
\begin{itemize}
  \item \textit{Because the join() method is in 
the string class, instead of the list class?
I agree it looks funny.}  
\item \textit{The simplest solution is to use
shell\_exec() to run the mysql client with the SQL script as input. 
This might run a little slower because it has to fork, but you can write 
the code in a couple of minutes and then get back to working on something useful. 
Writing a PHP script to run any SQL script could take you weeks\ldots.}
\end{itemize}
\normalsize
But in most of the cases he visits general software or coding questions
that are specifically related to databases and this fact is picked up by our
model and it assigns him predominantly (>0.8) a database role even though he 
is active contributor to
general software and coding discussions. MMSB on the other hand assins him 30\%
database (role 1) 30\% general coding (role 2) and rest is distributed among the
remaining 18 roles. Similarly Graclus assigns the user role 1 as 35\%, role 2
\25\% and remaining are divided almost equally to the rest of the roles. 

The model picks up other similar cases for which it is able
to successfully distinguish (compared to MMSB) between user's global and local
roles even though they are dynamic in nature.
% E.g.
% user\ldots~\comment{Write about the top 20 and bottom 20 users with most and
% least variations accross threads }. 

% \comment{Put Block matrix to demonstrate better resolution of cluster}\\
% \comment{Put nice visualization for some per user-thread topic proportions.}\\

\vspace*{-0.5\baselineskip}
\paragraph{Synthetic dataset: stability and scalability}
Figure~\ref{fig:syntheticRMSE} gives the rmse of the model for the recovery of
community topic $\pi$ over the synthetic dataset. From the graph, higher values
of the parameters make it harder to recover the $pi$ values. For this experiment
we fix topic number at 5 and vary $\alpha$ or $\eta$ by fixing the other at
0.01. The other priors such as $\kappa, \theta,
\omega$ etc. are fixed at the values used to generate the dataset.  
It is apparent that the rmse is more sensitive towards $\alpha$ values and
recovers them well compared to $\eta$. The results are averaged over 20 random
runs of the experiment for the given values of $\alpha$ and $\eta$. The rmse
achieved for lower values of priors $\alpha$ and $\eta$ is very promising as
it means that the confidence interval of the estimate is very high for sparse
priors for this model given sufficient data. 
Figure~\ref{fig:syntheticScalability} shows the scalability plot for two cases:
1) varying number users with fixed number for threads at 100, and 2) varying number of 
threads with fixed number of users at 1000. We observe that in both cases 
the model scale linearly with increasing threads or users. It is highly scalable 
even when the number users or threads approach 10 million. 

%\abhi{I will put in more time plots on paralleization for synthetic data with 
%varying number of users, threads, topics, words. Also plots from section
%4 about scalability will have to connected with this section.}

\section{Related Work}

Ho et al.
\cite{Ho:2012:DHT:2187836.2187936} provide TopicBlock that combines text and
network data in an efficient way for building a taxonomy for a corpus.
The LDA model and MMSB models were combined by
Nallapati et al. \cite{Nallapati:2008:JLT:1401890.1401957} using the
Pairwise-Link-LDA and Link-PLSA-LDA models where documents are assigned
membership probabilities into bins obtained by topic-models. These works 
combine MMSB and LDA to bring network and text together similar to us 
although for a tagential task. 
For simultaneously modeling topics in bilingual-corpora, Smet et al.
\cite{Smet:2011:KTA:2017863.2017915} propose the Bi-LDA model that generates
topics from the target languages for paired documents in these very languages.
They propose a latent space model that combines two LDA models 
where the end-goal is to classify any document into one of the
obtained set of topics. All the above works combine basic latent space modeling
blocks in interesting ways to provide useful results.  

White et al.\cite{ICWSM124638} propose a mixed-membership model that obtains
membership probabilities for discussion-forum users for each statistic
(in- and out-degrees, initiation rate and reciprocity) in various profiles and
clusters the users into ``extreme profiles'' for user role-identification
and clustering based on roles in online communities but this approach
doesnt take into account the text used by users.
Sachan et
al.~\cite{Sachan:2012:UCI:2187836.2187882} provide a model for community
discovery that combines network edges with hashtags and other heterogeneous data
and use it to discover communities in twitter and Enron email dataset. But they
do not consider the structure of the interaction into account as well they are 
not scalable. The largest dataset has less than 8500 users.
For modeling the behavioral aspects of entities and
discovering communities in social networks, several game-theoretic approaches
have been proposed (Chen et al. \cite{Chen:2010:GFI:1842547.1842566}, Yadati and
Narayanam \cite{Yadati:2011:GTM:1963192.1963316}) but none of them applies it to
a forum data containing both network and text. Zhu et
al.~\cite{Zhu:getoor:MMSB-text} combine MMSB and text for link prediction and
scale it to 44K links.
Ho et al.~\cite{HoYX12} provide  unique triangulated sampling schemes for scaling
mixed membership stochastic block models~\cite{Airoldi:2008:MMS:1390681.1442798} to
the order of hundreds of thousands of users. Prem et
al.~\cite{conf/nips/GopalanMGFB12} use stochastic variational inference 
coupled with sub-sampling techniques to
scale MMSB like models to hundreds of thousands of users.


None of the works above address the sub-network dynamics of thread based
discussion in online forums. Our work is unique in this context and tries to
bring user role modelling in online social networks closer to the
ground realities of online forum interactions.
Active sub-network modelling has been used recently to model gene interaction 
networks~\cite{Lichtenstein:Charleston}. They
combine gene expression data with network topology to provide bio-molecular 
sub-networks, though their approach is not scalable as they use simple EM for
their inference. We leverage the scalable aspects of
SVI~\cite{Hoffman:2013:SVI} to combine MMSB (network topology) with LDA (post
contents) in a specific graphical structure (thread structure in the forum) to
obtain a highly scalable active sub-network discovery model.

Matrix factorization and spectral learning based approaches are some of the
other popular schemes for modelling user networks and content. In recent
past both approaches have been made scalable to a million order node size graph
~ \cite{Gemulla:2011:LMF,Dhillon:2005}. But these methods are unable to incorporate 
the rich structure that a probabilistic modeling based method 
can take into account as shown by the empirical results in the 
previous section.



\section{Discussion and Future Work}
The  proposed model relies on the fact that forum users have dynamic role
assignments in online discussions and leveraging this fact helps to increase
prediction performance as well as understand the forum activity. The model
performs very well in its prediction tasks. It outperforms all the other methods
over all the datsets by a huge margin. The model is scalable and is able to run
on social network dataset of unprecedented content size. There is no past
research work that scales forum contents to more than one  million user and
around 10 million posts. 

The idea that active subnetwork is useful in modelling online forums is
demonstrated qualitatively and quantitatively. Quantitatively it provides better
prediction performance and qualitatively it captures the dynamics of user
roles in forums. This dynamism helps us find new user roles that may have
been missed by state of the art clustering approaches.
From the synthetic experiments it is observed that the model
recovers its parameters with high likelihood with sparse priors. This works to
its advantage for scalable learning as big data sets tend to be sparse.

The scalability aspects of the inference scheme proposed here are worth noting.
Besides the multi-core and stochastic sub-sampling components of the proposed
inference, the use of Poisson to model the edge weights has enabled us to 
ignore zero-edges if need be. This reduces the amount of work needed for
learning the network parameters. The learned network is at par with the state of the art
inference schemes as demonstrated in the prediction tasks.   

One aspect to explore in future is to combine multiple types of links in
network. For example in many online forums users explicitly friend other users,
follow other users or are members of same forum related sub-groups as other
users. All these relations can be modelled as a graph. It is worth finding out
how important is modelling active sub-network in such a case. It is
possible that various types of links might reinforce each other in learning
the parameters and thus will obviate the need to model a computationally
costly sub-network aspect. As we saw in figure~\ref{fig:syntheticRMSE} that
sparsity helps, but how sparser can we get before we start getting poor results
needs some exploration.

As we have
seen, figure ~\ref{fig:syntheticRMSE}, that the model recovers the
community-topic parameters with very high likelihood for lower values of model
priors $\alpha$ and $\tau$. If this is a general attribute of active sub-network
models then it can be leveraged for sparse learning. Moreover, although in case
of large online forums modelling active sub-networks is computationally
challenging and costly, the sparsity aspects of active sub-networks might help
reduce the computation costs.


\bibliography{mybib}
\bibliographystyle{plain}
\appendix
\label{sec:appendix}
% \onecolumn{}

%The log-likelihood of the model:
%\begin{align}
%\log L &= \log \! P(Y, W, Z_{\leftarrow}, 
%Z_{\rightarrow}, \Pi, B, \beta | \alpha, \eta, \theta, \alpha) \nonumber\\
%\nonumber &= \sum_{t} \bigg[ \sum_{p,q} \! \log P(Y_{t,p, q} | Z_{t,p \rightarrow q} 
%     , Z_{t,p \leftarrow q}, B) \\ \nonumber 
%     &+ \sum_{p,q} \log P(Z_{t, p \rightarrow q} | \Pi_q) \\ \nonumber 
%     & + \sum_{p,q} \log \! P(Z_{t, p \leftarrow q} | \Pi_{q}) \bigg] 
%     + \sum_{p} \log \! P(\Pi_{p} | \alpha)  \\ \nonumber 
%     & + \bigg[ \sum_{t=1}^{T} \! \sum_{p \in t} \sum_{i=1}^{N_{T_{p}}} 
%     \log \! P(w_{t,p,i} | Z'_{t,p,i}, \beta) \\ \nonumber 
%     & + \sum_{t=1}^{T} \sum_{p \in t} \sum_{i=1^{N_{T_{p}}}} \log \! 
%     P(Z'_{t,p,i} | \bar{Z}_{t, p \rightarrow q}) \bigg]   
%     \\ \nonumber & + \sum_{k} \log P(\beta_{k} | 
%     \eta) +  \sum_{g,h} \log P(B_{g,h} | \kappa, \theta).\\ 
%     \label{eqn:LL}
%\end{align}

%The data likelihood for the model in figure~1
%
%\begin{align}
%P(Y, R_{p} | \alpha, \beta, \kappa, \eta) &=  \nonumber\\ 
% \int_{\Phi} \!
%\int_{\Pi} \sum_{z} \! P(Y, R_{p}, & z_{p \rightarrow q}, z_{p \leftarrow q},
%\Phi, \Pi | \alpha, \beta, \kappa, \eta)  \nonumber \\  \nonumber
%%\\ 
%= \int_{\Phi} \! \int_{\Pi} \sum_{z} \! \bigg[ \prod_{p,q} & \prod_{t}
%P(Y_{pq}^{t} | z_{p \rightarrow q}^{t}, z_{p \leftarrow q}^{t}, B) 
%\cdot P(z_{p \rightarrow q}^{t} | \Pi_{p}) \nonumber
%\\  \cdot P(z_{p \leftarrow q}^{t} |
%\Pi_{q})   & \cdot \bigg(\prod_{p} P(\Pi_{p} | \alpha) \prod_{t} \prod_{p}
%P(R_{p}^{t} | z_{p \rightarrow q}^{t}, \Phi) \nonumber
%\\ \cdot \prod_{k} P(\Phi_{k} |
%\beta)&\bigg) \cdot \prod_{g,h}P(B_{gh} | \eta, \kappa) \bigg].
%\end{align}

%The complete log likelihood of the model is:
%
%\begin{align}
%\log \! &P(Y, W, z_{\rightarrow}, z_{\leftarrow}, \Phi, \Pi, B | \kappa, \eta,
%\beta, \alpha) \nonumber
%\\ & = \sum_{t} \! \sum_{p,q} \! \log P(Y_{pq}^{t} | z_{p
%\rightarrow q}^{t} , z_{p \leftarrow q}^{t}, B) \nonumber
%\\ &+ \nonumber \sum_{t} \!
%\sum_{p,q} \! (\log P(z_{p \rightarrow q}^{t} | \Pi_{p}) + \log \! P(z_{p \leftarrow q}^{t} |
%\Pi_{p})) \\  
%&+ \sum_{p} \! \log \! P(\Pi_{p} | \alpha) ~+ \sum_{t} \!
%\sum_{p} \! \sum_{w \in R_{p}^{t}} \log P(w | z_{p \rightarrow}, \Phi)
%\nonumber\\ 
%&+ \sum_{k} \! \log P(\Phi_{k} | \beta) + \sum_{gh} \! \log P(B_{gh} | \eta,
%\kappa).
%\end{align}

%The mean field variational approximation for the posterior is 
%
%\begin{align}
%q(z, &\Phi, \Pi, B | \Delta_{z_{\rightarrow}}, \Delta_{\Phi}, \Delta_{B},
%\Delta_{z_{\leftarrow}}, \Delta_{B_{\kappa}})  = \nonumber \\ \prod_{t} \!
%& \prod_{p,q} \! \bigg( q_{1}(z_{p \rightarrow q}^{t} | \Delta_{z_{p \rightarrow
%q}}) + q_{1}(z_{p \leftarrow q}^{t} | \Delta_{z_{p \leftarrow q}})  \bigg) \nonumber \\
%\cdot \prod_{p} &\! q_{4}(\Pi_{p} | \Delta_{\Pi_{p}}) \prod_{k} q_{3} (\Phi_{k}
%| \Delta_{\Phi_{k}}) \prod_{g,h} \! q(B_{g,h} | \Delta_{B_{\eta}},
%\Delta_{B_{\kappa}}).
%\end{align}

The lower bound for the data log-likelihood from jensen's inequality is: 

\begin{align}
&L_{\Delta} = E_{q}\bigg[ \log \! P(Y, W, z_{\rightarrow}, z_{\leftarrow}, \Phi,
\Pi, B | \kappa, \eta, \beta, \alpha) - \log \! q \bigg]\nonumber\\
&= E_{q} \Bigg[ \sum_{t} \! \sum_{p,q} \! \log \left(
B_{g,h}^{Y_{p,q}^t} \frac{e^{-B_{gh}}}{Y_{pq}^{t}!} \right) +
\sum_{t} \! \sum_{pq} \! \log\left( \prod_{k} (\pi_{p,k}^{z_{p \rightarrow q} =
k}) \right) \nonumber\\
&+ \sum_{t} \! \sum_{p,q} \log \! \left(
\prod_{k}(\pi_{q,k})^{z_{p \leftarrow q} = k} \right)\nonumber\\ 
&+\sum_{t} \! \sum_{p} \! \sum_{w\in R_p^t}  \log \! \left(
\prod_{u\in V}(\bar{z}^T\phi_u)^{w = u} \right)
\nonumber\\ &+ 
\sum_{p} \! \log \left( \prod_{k}
(\Pi_{p,k})^{\alpha_{k} - 1} \cdot \frac{\Gamma(\sum \alpha_{k})}{\prod_{k}
\Gamma(\alpha_{k})} \right) \nonumber\\ & + 
\sum_{k} \! \log\left( \prod_{u\in V}
(\phi_{k,u})^{\beta_{k} - 1} \cdot \frac{\Gamma(\sum \beta_{k})}{\prod_{k}
\Gamma(\beta_{k})} \right) \nonumber\\ &+
 \sum_{g,h} \! \log \! \left( B_{g,h}^{\kappa - 1} /
\eta^{\kappa} \Gamma(\kappa) \cdot \exp(-B_{g,h}/\eta) \right) \Bigg]
\nonumber\\ 
& -E_{q} \Bigg[ \sum_{t} \! \sum_{p,q} \log \big( \prod_{k} (\Delta_{z_{p
\rightarrow q}, k})^{z_{p \rightarrow q}=k} \big) \nonumber \\&+ \sum_{t} \!
\sum_{p,q} \! \log \! \left(
\prod_{k} \! (\Delta_{z_{p \leftarrow q}, k})^{z_{p \leftarrow q} = k} \right)
  \nonumber \\
 &+\sum \! \log \left( \prod_{k} \! (\Pi_{p,k})^{\Delta_{\pi_{pk}}-1}
\frac{\Gamma(\Delta_{\Pi_{p}})}{\prod_{k=1} \! \Gamma(\Delta_{\Pi_{p,k}})}
\right) \nonumber \\ &+ 
\sum_{k} \log \! \left( \prod_{u \in v}
(\Phi_{k,u})^{\Delta_{\Phi_{ku}} - 1)} \frac{\Gamma(\Delta_{\Phi_{k}})}
{\prod_{u \in v} \! \Gamma(\Delta_{\Phi_{k,u}})} \right) \nonumber \\ 
&+ \sum_{g,h} \log \! \left(
\frac{B_{g,h}^{\Delta_{\kappa = 1}}}{\Delta_{\eta}^{\Delta_{\kappa}}
\Gamma(\Delta_{\kappa})} \exp(-B_{g,h}/\Delta_{\eta}) \right) \Bigg].
\label{eqn:VarLowerBound}
\end{align}

%$\Delta_{\phi}$ used in the update of $\phi$ in equation~\ref{eqn:phiUp}. The
%parameter $\omega$ is used here to balance out the contribution from the text
%side to the network. 
%
%\begin{align}
%\Delta_{\phi^{'}_{t,p,g,h}} &= y_{t,p,q}( \log \! \lambda_{g,h} + 
%\Psi(\nu_{g,h})) - \nu_{g,h} \lambda_{g,h} - \log \! (y_{t,p,q}!)
%\nonumber \\ & + \Psi(\gamma_{p,g}) - \Psi(\sum_{g} \gamma_{p,g})
%\nonumber \\  & + \Psi(\gamma_{q,h}) - \Psi(\sum_{h} \gamma_{q,h})
%\nonumber \\  & + \omega\sum_{i=1}^{N_{T_{P}}} \! \chi_{t,p,i,g} 
%\bigg[ \ln \! \frac{\epsilon}{\delta_{p,t}} - \frac{1}{\delta_{t,p}} 
%+ \ln \bigg( 1 + \frac{\epsilon}{\delta_{p,t}} \bigg) 
%\cdot \frac{1}{\delta_{t,p}} \bigg].
%\label{eqn:phiDelta}
%\end{align}
%
%$\Delta_{\chi}$ used in equation~\ref{eqn:chiUp} for $\chi$ update
%
%\begin{align}
%\Delta_{\chi^{'}_{t,p,g,h}} &= \bigg[ \Psi(\tau_{k,w_{t,p,i}}) - 
%\Psi(\sum_{w_{t,p,i}} \! \tau_{k, w_{t,p,i}}) \bigg]
%\nonumber \\  & + \ln \! (\frac{\epsilon}{\delta{p,t}}) \frac{1 - 
%\sum_{q,h} \! \phi_{t,p,q,k,j}}{\delta_{t,p}} 
%\nonumber\\  & + \frac{\sum_{q,h} \phi_{t,p,q,k,h}}{\delta_{t,p}} 
%\ln \! (1 + \frac{\epsilon}{\delta_{t,p}}).
%%\nonumber\\  & - 1 - \log \! \chi_{t,p,i,k}
%\label{eqn:chiDelta}
%\end{align}


Partial derivative of $\nu$
\begin{align}
\frac{dL}{\partial\nu_{g,h}} &= \sum_{t} \! \sum_{p,q} \! 
\phi_{t,p,q,g,h} (y_{t,p,q} \Psi'(\nu_{g,h}) - \lambda_{g,h})
\nonumber\\  & + (\kappa_{g,h} - \nu_{g,h})\Psi'(\nu_{g,h}) + 
1 - \frac{\lambda_{g,h}}{\theta_{g,h}}.
\label{eqn:partialNu}
\end{align}

The traditional variational updates for the global parameters

\begin{align}
\gamma_{p,k} &= \alpha_{k} + \sum_{t} \! \sum_{q} \! \sum_{h} \! \phi_{t,p,q,k,h} 
+ \sum_{t} \! \sum_{q} \! \sum_{g} \! \phi_{t,q,p,g,k}.
\label{eqn:gammaUp}
\end{align}

\begin{align}
\nu_{g,h}^{t+1} &= \nu_{g,h}^{t}+\rho_\nu \frac{dL}{\partial\nu_{g,h}}. 
\label{eqn:nuUp}
\end{align}

\begin{align}
\lambda_{g,h} &= \frac{\bigg( \sum_{t} \! \sum_{p,q} \! \phi_{t,p,q,g,h} y_{t,p,q} + 
\kappa_{g,h} \bigg) }{
 \bigg( \bigg( \sum_{t} \! \sum_{p,q} \! \phi_{t,p,q,g,h} \bigg) + 
\frac{1}{\theta_{g,h}} \bigg) \nu_{g,h}}.
\label{eqn:lambdaUp}
\end{align}

\begin{align}
\tau_{p,v} = \nu_{v} + \sum_{t} \! \sum_{p \in t}
\bigg(\sum_{w_{t,p,i}=v}^{N_{t,p}} \chi_{t,p,i,k} \bigg).
\label{eqn:tauUp}
\end{align}


where $\rho_\nu$ is $\nu$'s gradient ascent update step-size using its partial
derivative $\frac{dL}{\partial\nu_{g,h}}$ define in equation~\ref{eqn:partialNu}.



\end{document}
