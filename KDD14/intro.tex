Online forums are a microcosm of communities where users' presentation
characteristics vary across different parts of the forum. Users participate in a
discussion or group activity by posting on a related thread. And during his
stay in a forum, a user participates in many different discussions and posts on multiple
threads. The thread level presentation characteristics of a user are different
than the global presentation chracteristics. A participating user gears his
responses to suit specific discussions on different threads. These thread based
interactions give rise to active sub-networks, within the global network of users,
that characterize the dynamics of interaction. Overlaying differential changes
in user interaction characteristics across these sub-networks provides
insights into users' macroscopic (forum-wide) as well as microscopic (thread
specific) participation behavior. 

Analysing online social networks and user forums have been approached using
various perspective such as graph/network ~\cite{Shi:2000:NCI:351581.351611,
Shi00learningsegmentation} , probabilistic 
graphical model~\cite{ Airoldi:2008:MMS:1390681.1442798}, 
combined network \& text mining
based~\cite{Ho:2012:DHT:2187836.2187936,Nallapati:2008:JLT:1401890.1401957}
based approaches. But none of the approches above in social networks have taken
into account the dynamics of sub-networks and the related thread-based structural framework in
which the discussions in online forums happen. Whereas active-subnetwrok modelling has
been very useful to the research in computational biology in recent years where
it's been used to model sub-network of gene
interactions~\cite{journals/ploscb/DeshpandeSVHM10,Lichtenstein:Charleston},
very few approaches using active-subnetwork have been proposed to model online
forum user interactions. Taking into account subnetwork interaction
dynamics is important to correctly model the user particiaption behavior. E.g.
in an onlne forum there are topic-threads and users post their responses on
these threads after possibly reading through the responses of other users in
these threads. The users possibly post multiple times on the thread in the form
of replies to other posts in the thread. For analysing such a user interaction it
becomes imperative that the structure of the conversation must also be taken
into account  besides taking into account the user interaction network and the
text posted. This enables us to gain deeper insights into user behavior in the
online community that was not possible earlier. 

One of the main challenges of this work has been the ability to deal with
social network data on a large (millions of users and threads) scale. A
social network spanning around millions of users and threads would be an ideal
case to demonstrate effectiveness of active-subnetwprk modelling. To this
purpose we designed a model based on Stochastic variational 

The model also incorporates strength of interaction
among the users by incorporating interaction counts as compared to MMSB model
which just looks presence or absence of link~\cite{Airoldi:2008:MMS:1390681.1442798}. 
In the process we discover interesting online communities and social phenomena.

The current work also focuses on analysing large scale user interactions in big
online social forums. We provide a stochastic variational
approximation~\cite{Hoffman:2013:SVI} based estimation technique that is
scalable to big forums with thousands of users.

% ~\comment{Also write about stochastic approximation if given we have time and we
% can get it working}. 


