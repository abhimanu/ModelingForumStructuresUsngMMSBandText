We analyse three different forums: 1) Cancer forum, 2) Stack Overflow, and
3) Reddit.
The three datasets mentioned above represent three different sets of
online gatherings which helps us genearlize pur claims. 
% \subsection{Wikipedia talk pages}
% Wikipedia currently hosts more than four million articles on a wide range of topics.
% Quality control on Wikipedia occurs through discussions on the Wikipedia talk pages. 
% Every article on Wikipedia has a corresponding talk-page. Contributors to Wikipedia 
% discuss edits by other users, topics that can be used to extend the article, 
% the veracity of the article's contents etc. Talk-pages provide functionality
% for threaded discussions that are used as dialog among users. This rich
% structured discussion manifests itself as a social network that can be mined and
% studied.  A standard Wikipedia talk page consists of topics which hold
% discussion threads. For building our dataset, we used a snapshot of Wikipedia on
% the 1st of October 2012 \cite{wikipedia}. We built a parser and extracted the
% thread structure in the talk-pages to build the matrices. There are 20,000 users
% in our datasets that span accross 30,000 talk pages ~(\comment{These figures
% will change depending on whether we want to incorporate more or less users in
% future}).
% The talk pages become the threads in context of our graphical model.

\subsection{Cancer Forum}
The cancer forum is an online forum where users discuss about their cancer
treatment and any thing else under the sun. Here again the conversation happens
in a structured way where users post their responses on a thread by thread
basis. Users also call each other by their names (or nick-names) while posting
in many cases. This forum has around 3000 users and 10,000 threads, and a user
on average posts around 120 words in a post.

\subsection{Stack Overflow}
This is an online forum where users ask and answer technical troubles. It is a
typical online forum where users reply to each other in a threaded structure.
Based on the response the replies are voted up and down by other users. This
voting score is used in our prediction tasks later. \comment{Shriphani will
provide the exact statistics of this dataset as soon as the crawl is done. 
We will have most likely have around 1/2 a million users in this set.}


\subsection{Reddit}
Reddit is an online trend spotting website where users post interesting
articles, news, stories, links etc. and a discussion ensues. Users can upvote or
downvote any reply or posts. The converstaions happen in a threaded structure as
described in our generative story. The upvotes are used later by our model for
prediction task. \comment{This is again an ongoing crawl but
we should have atleast 200K+ users here. Shriphani please provide the latest
numbers soon}

% \comment{Need to expand more in the dataset section with more numbers and
% stats. Though the stats would be much clearer as and when we perform the
% experiments}
