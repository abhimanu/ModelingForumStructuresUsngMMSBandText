Our project lies at the intersection of graph clustering and social emergence.
Wikipedia's talk pages are an instance of a large social community where we can
observe social emergence. Sociologists divide emergents into various
levels \cite{keith} e.g. Individual-specific interactions, Ephemeral emergent
interactions, Stable emergent interaction etc.

For role-identification and clustering users based on roles in online communities, 
White et al.\cite{ICWSM124638} proposed a mixed-membership model that obtained
membership probabilities for discussion-forum users for each statistic
(in- and out-degrees, initiation rate and reciprocity) in various profiles and 
clustered the users into ``extreme profiles''. Ho et al.
\cite{Ho:2012:DHT:2187836.2187936} presented TopicBlock that combines text and 
network data for building a taxonomy
for a corpus. 

Griffiths et al. \cite{griffiths2004finding} described a generative model for 
Blei et al.'s LDA model \cite{blei2003latent} using an MCMC algorithm for 
bayesian inference queries on the model.

The LDA model and MMSB models were combined by
Nallapati et al. \cite{Nallapati:2008:JLT:1401890.1401957} using the Pairwise-Link-LDA
and Link-PLSA-LDA models where documents are assigned membership probabilities into
bins obtained by topic-models.

For simultaneously modeling topics in bilingual-corpora, Smet et al.
\cite{Smet:2011:KTA:2017863.2017915} proposed the Bi-LDA model which generates
topics from the target languages for paired documents in these very languages.
The end-goal of their approach is to classify any document to one of the
obtained set of topics.

For modeling the behavioral aspects of entities and discovering communities in 
social networks, several game-theoretic approaches have been proposed 
(Chen et al. \cite{Chen:2010:GFI:1842547.1842566}, Yadati and Narayanam
\cite{Yadati:2011:GTM:1963192.1963316}).

Our work is unique in this context as it tries to bridge the gap between 
community discovery and social emergence. A very popular approach for network
clustering is to use a generative framework to infer the underlying structure of
communities in a graph. Airoldi et al.\cite{Airoldi:2008:MMS:1390681.1442798}
proposed a mixed membership stochastic block model to infer community structure
in a network. Here they use a generative scheme which captures membership of a
user to multiple communities using latent variables. There are other
variants of mixed membership stochastic models, e.g. Ho et
al.\cite{HoSonXin11} that study evolving clusters over time varying networks while Fu
et al.\cite{Fu:2009:DMM:1553374.1553416} explore dynamic mixed membership models.
