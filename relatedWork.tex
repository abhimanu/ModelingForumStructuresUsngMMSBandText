Our project lies at the intersection of community discovery, structure modelling
and text mining. Wikipedia's talk pages are an instance of a large social
community where we can observe users networking with each other as well as
posting content in a structured way. Similar phenomena are observed across
almost all social networking websites and online forums.


For role-identification and clustering users based on roles in online communities, 
White et al.\cite{ICWSM124638} proposed a mixed-membership model that obtained
membership probabilities for discussion-forum users for each statistic
(in- and out-degrees, initiation rate and reciprocity) in various profiles and 
clustered the users into ``extreme profiles''. In a similar work, Ho et al.
\cite{Ho:2012:DHT:2187836.2187936} presented TopicBlock that combines text and 
network data for building a taxonomy for a corpus. 
The LDA model and MMSB models were combined by
Nallapati et al. \cite{Nallapati:2008:JLT:1401890.1401957} using the
Pairwise-Link-LDA and Link-PLSA-LDA models where documents are assigned
membership probabilities into bins obtained by topic-models.

For simultaneously modeling topics in bilingual-corpora, Smet et al.
\cite{Smet:2011:KTA:2017863.2017915} proposed the Bi-LDA model that generates
topics from the target languages for paired documents in these very languages.
The end-goal of their approach is to classify any document into one of the
obtained set of topics. For modeling the behavioral aspects of entities and
discovering communities in social networks, several game-theoretic approaches
have been proposed (Chen et al. \cite{Chen:2010:GFI:1842547.1842566}, Yadati and
Narayanam \cite{Yadati:2011:GTM:1963192.1963316}). Zhu et
al.~\cite{Zhu:getoor:MMSB-text} combine MMSB and text for link prediction and
scal it to 44K links.

Ho et al.~\cite{HoYX12} provide a unique triangulated sampling schemes for scaling
mixed membership stochastic block models~\cite{Airoldi:2008:MMS:1390681.1442798} to
hundreds of thousands users based communities. Prem et al.~\cite{conf/nips/GopalanMGFB12}
use stochastic variational inference coupled with sub-sampling techniques to
scale MMSB like models to hundreds of thousands of users.

None of the works above address the strucutre of the information flow in an
online community. Our work is unique in this context as it tries to bridge the
gap between community discovery and strucured interaction. We propose a novel
modelling scheme that takes into account the network information, user contents
as well as structure of the interaction and is scalable to big online forums.
