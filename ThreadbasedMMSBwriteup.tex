\documentclass{article}
\usepackage{nips12submit_e,times}

\usepackage{amsmath}
\usepackage{amsfonts}
\usepackage{amssymb}
\usepackage{graphicx, subfigure, fink, grffile, placeins}
\usepackage{hyperref}
\usepackage{color}
\usepackage{url}
\usepackage{algorithm}
\usepackage{algpseudocode}
\usepackage{upgreek}

% \title{Modeling Network and Text Structure in Online Forums}
% \title{Simultaneous text \& Network modeling in MMSB for Learning Structured
% Interactions on Online Forums}
% \title{Modeling Structured Interaction in Big Online User Forums }
\title{Modeling Structured Interaction in \\ Large Online User Forums }

% \author{
% Abhimanu Kumar \\
% Carnegie Mellon University\\
% \texttt{abhimank@cs.cmu.edu}\\
% \And
% Chong Wang \\
% Carnegie Mellon University\\
% \texttt{chongw@cs.cmu.edu}\\
% \And
% Carolyn P. Rose \\
% Carnegie Mellon University\\
% \texttt{cprose@cs.cmu.edu}\\
% }

\newcommand{\fix}{\marginpar{FIX}}
\newcommand{\new}{\marginpar{NEW}}
\newcommand{\comment}[1]{{\color{red}{#1}}}

%\nipsfinalcopy

\begin{document}
\maketitle
\begin{abstract}
We present here an approach to model online social forums that respects the
structure of the discussion and thus inturn provides researcher with unique
insights into the myriads of user forums in the todays online communties. We
bring together the structure of the forum network as well as texts posted
in a model that respects the structure of the conversation in the forum. The
work also focuses on large scal interaction and provides an efficient
approximate estimation technique that is scalable~\comment{i am thinking of
using stochastic variational for scalability if time permits} .
Analysing Wikipidia edit and cancer patient online user forums using this 
technique provides us interesting
insights into these online user communities.
\end{abstract}

\section{Introduction}
We discuss here the ideas and questions that are important for modeling
structure in online forums. Following are the questions that we want answered:
\begin{enumerate}
  \item When two persons interact in a thread or a post which topic/community they
each belong to
	\item When a user U is in communnity C what type of text does he use to
	communicate
	\item Multi-user-Interaction: In a thread a user can post by addressing to a
	specific user but he is also talking to other users in the thread simultaneously. Can we model this
	phenomenon
	\item There is an inherent bias towards the thread starter or in turn topic of
	the theread; can such an information be utilised in some form of a prior
	value/input
	\item Multi-layer-Interaction: On the network side of things there are multiple
	signals which cannt be simply added to make a single signal e.g. different
	types of edges in the graphs (user calling by username and nick-name). Can the
	model take this into account. We are not doing this at present.
	\item User posts aggregation; there are multiple ways to aggreagte
	\begin{enumerate}
		\item Network Layer aggregation: We call all types of edges as a single
		edge type and use this combined signal. 
		\item aggregating user posts accross multiple threads in the forum.
		\item Aggregating user post only in the same thread
		\item Aggregating user post only for same user-user pair interaction; i.e. a user might have
		posted multiple replies to another user and we aggregate all such replies into
		one for this user pair interaction.
		\item No aggregation at all.
	\end{enumerate}
\end{enumerate}

\section{Related Work}
Our project lies at the intersection of graph clustering and social emergence.
Wikipedia's talk pages are an instance of a large social community where we can
observe social emergence. Sociologists divide emergents into various
levels \cite{keith} e.g. Individual-specific interactions, Ephemeral emergent
interactions, Stable emergent interaction etc.

For role-identification and clustering users based on roles in online communities, 
White et al.\cite{ICWSM124638} proposed a mixed-membership model that obtained
membership probabilities for discussion-forum users for each statistic
(in- and out-degrees, initiation rate and reciprocity) in various profiles and 
clustered the users into ``extreme profiles''. Ho et al.
\cite{Ho:2012:DHT:2187836.2187936} presented TopicBlock that combines text and 
network data for building a taxonomy
for a corpus. 

Griffiths et al. \cite{griffiths2004finding} described a generative model for 
Blei et al.'s LDA model \cite{blei2003latent} using an MCMC algorithm for 
bayesian inference queries on the model.

The LDA model and MMSB models were combined by
Nallapati et al. \cite{Nallapati:2008:JLT:1401890.1401957} using the Pairwise-Link-LDA
and Link-PLSA-LDA models where documents are assigned membership probabilities into
bins obtained by topic-models.

For simultaneously modeling topics in bilingual-corpora, Smet et al.
\cite{Smet:2011:KTA:2017863.2017915} proposed the Bi-LDA model which generates
topics from the target languages for paired documents in these very languages.
The end-goal of their approach is to classify any document to one of the
obtained set of topics.

For modeling the behavioral aspects of entities and discovering communities in 
social networks, several game-theoretic approaches have been proposed 
(Chen et al. \cite{Chen:2010:GFI:1842547.1842566}, Yadati and Narayanam
\cite{Yadati:2011:GTM:1963192.1963316}).

Our work is unique in this context as it tries to bridge the gap between 
community discovery and social emergence. A very popular approach for network
clustering is to use a generative framework to infer the underlying structure of
communities in a graph. Airoldi et al.\cite{Airoldi:2008:MMS:1390681.1442798}
proposed a mixed membership stochastic block model to infer community structure
in a network. Here they use a generative scheme which captures membership of a
user to multiple communities using latent variables. There are other
variants of mixed membership stochastic models, e.g. Ho et
al.\cite{HoSonXin11} that study evolving clusters over time varying networks while Fu
et al.\cite{Fu:2009:DMM:1553374.1553416} explore dynamic mixed membership models.


% \section{Graphical Model \& Generative Story}
\section{Approach}
Online forums have a generally a specific structure that provides a lot of
context to all the interactions among the users. Ignoring this in the analysis
makes us lose a lot of precious information as we will see in later sections.
Here we describe a typeical forum and the answers that we plan to obtain.

\subsection{structure in online forums}
We discuss here the ideas and questions that are important for modeling
structure in online forums. Following are the questions that we want answered:
\begin{enumerate}
  \item When two persons interact in a thread or a post which topic/community they
each belong to
	\item When a user U is in communnity C what type of text does he use to
	communicate
	\item Multi-user-Interaction: In a thread a user can post by addressing to a
	specific user but he is also talking to other users in the thread simultaneously. Can we model this
	phenomenon
	\item There is an inherent bias towards the thread starter or in turn topic of
	the theread; can such an information be utilised in some form of a prior
	value/input
	\item Multi-layer-Interaction: On the network side of things there are multiple
	signals which cannt be simply added to make a single signal e.g. different
	types of edges in the graphs (user calling by username and nick-name). Can the
	model take this into account. We are not doing this at present.
	\item User posts aggregation; there are multiple ways to aggreagte
	\begin{enumerate}
		\item Network Layer aggregation: We call all types of edges as a single
		edge type and use this combined signal. 
		\item aggregating user posts accross multiple threads in the forum.
		\item Aggregating user post only in the same thread
		\item Aggregating user post only for same user-user pair interaction; i.e. a user might have
		posted multiple replies to another user and we aggregate all such replies into
		one for this user pair interaction.
		\item No aggregation at all.
	\end{enumerate}
\end{enumerate}

\subsection{graphical model \& generative story}
Based on the discussions above we came up with the following final model shown
in figure~1. In this model,
figure~1 below, we aggregate the posts of 
a given user in a given thread into one document called $R_p$. 

The generative process for the figure is as follows:

\begin{figure}
\includegraphics[0.3\textwidth]{pgm_ThreadBased.png}
\label{fig:finalThreadAggregationModel}
\caption{This graphical model takes into account multi-way interaction among
users in a thread simultaneously}
\end{figure}

Assuming that there are total $N_t$ users in the thread $t$.  
\begin{itemize}
  \item For each Thread $t$
\begin{itemize}
  \item For each user $p \in \mathcal{N}_t$
  \begin{itemize}
    \item Draw a $K$ dimensional mixed membership vector 
    $\overset{\rightarrow}{\uppi}_{p} \sim$ Dirichlet($\alpha$)

    \item Draw $B(g,h) \sim Gamma(\kappa,\eta)$; where $\kappa, \eta$ are
    parameters of the gamma distribution.
  \end{itemize}

  \item For each pair of users $(p, q) \in \mathcal{N}_t \times \mathcal{N}_t$:
  \begin{itemize}
    \item Draw membership indicator for the indicator, 
    $\overset{\rightarrow}{z}_{(p \rightarrow q,t)} \sim$
    Multinomial($\uppi_{p}$).
    \item Draw membership indicator for the receiver,
    $\overset{\rightarrow}{z}_{(q \rightarrow p,t)} \sim$
    Multinomial($\uppi_{q}$).
    \item Sample the value of their interaction, $Y(p,q,t) \sim$
    Poisson(${\overset{\rightarrow}{z}}^{\top}_{(p \rightarrow q,t)}
    B~\overset{\rightarrow}{z}_{(p \leftarrow q,t)}$). 
%     We make the assumption that
%     two interactions are independent of each other i.e $Y(p,q,i)$ and $Y(p,q,j)$
%     are independent of each other where $i\neq j$.
	\end{itemize}
	\item For each user $p \in \mathcal{N}_t$
	\begin{itemize}
	  \item Draw $\phi_{k}$ from $Dirichlet(\beta)$.
	  \item Form the set $Q_{p,t}$ that contains all the users that p interacts to
	  on thread $t$
	  \begin{itemize}
	    \item For each word $w \in R_{p,t}$ 
	    \item Draw $w \sim \phi(w|z_{(p \rightarrow q,t)}, \forall q\in Q_{p,t})$  
	  \end{itemize}
% 	  \item Draw $W \sim $
%     \item Draw $z_{u,m}$ topic for user $U$'s document from $\pi_u$.
%     \item Draw $\tau_{k}$ from $Dirichlet(\beta)$.
%     \item Draw a word $w_{u,m}$ from $\tau_{z}$
  \end{itemize}
\end{itemize}  
\end{itemize}

The data likelihood for the model in figure~1

\begin{eqnarray}
P(Y, R_{p} | \alpha, \beta, \kappa, \eta) = \int_{\Phi} \! \int_{\Pi} \sum_{z} \! P(Y, R_{p}, z_{p \rightarrow q}, z_{p \leftarrow q}, \Phi, \Pi | 
\alpha, \beta, \kappa, \eta)  \nonumber \\  \nonumber
\\ = \int_{\Phi} \! \int_{\Pi} \sum_{z} \! \bigg[ \prod_{p,q} \prod_{t}
P(Y_{pq}^{t} | z_{p \rightarrow q}^{t}, z_{p \leftarrow q}^{t}, B) 
\cdot P(z_{p \rightarrow q}^{t} | \Pi_{p}) \cdot P(z_{p \leftarrow q}^{t} |
\Pi_{q})  \nonumber
\\ \cdot \left(\prod_{p} P(\Pi_{p} | \alpha) \prod_{t} \prod_{p} P(R_{p}^{t} |
z_{p \rightarrow q}^{t}, \Phi) \cdot \prod_{k} P(\Phi_{k} | \beta)\right) \cdot
\prod_{g,h}P(B_{gh} | \eta, \kappa) \bigg]
\end{eqnarray}

The complete log likeliood of the model is:

\begin{align}
\log \! P(Y, W, z_{\rightarrow}, z_{\leftarrow}, \Phi, \Pi, B | \kappa, \eta,
\beta, \alpha) = \sum_{t} \! \sum_{p,q} \! \log P(Y_{pq}^{t} | z_{p \rightarrow
q}^{t} , z_{p \leftarrow q}^{t}, B)~+ \nonumber  \\\nonumber \sum_{t} \!
\sum_{p,q} \! (\log P(z_{p \rightarrow q}^{t} | \Pi_{p}) + \log \! P(z_{p \leftarrow q}^{t} |
\Pi_{p})) + \sum_{p} \! \log \! P(\Pi_{p} | \alpha) ~+\\  \sum_{t} \!
\sum_{p} \! \sum_{w \in R_{p}^{t}} \log P(w | z_{p \rightarrow}, \Phi) +
\sum_{k} \! \log P(\Phi_{k} | \beta) + \sum_{gh} \! \log P(B_{gh} | \eta,
\kappa)
\end{align}

The mean field variational approximation for the posterior is 

\begin{align}
q(z, \Phi, \Pi, B | \Delta_{z_{\rightarrow}}, \Delta_{\Phi}, \Delta_{B},
\Delta_{z_{\leftarrow}}, \Delta_{B_{\kappa}}) = \prod_{t} \! \prod_{p,q} \!
\bigg( q_{1}(z_{p \rightarrow q}^{t} | \Delta_{z_{p \rightarrow q}}) +
q_{1}(z_{p \leftarrow q}^{t} | \Delta_{z_{p \leftarrow q}})  \bigg) \nonumber \\
\cdot \prod_{p} \! q_{4}(\Pi_{p} | \Delta_{\Pi_{p}}) \prod_{k} q_{3} (\Phi_{k} |
\Delta_{\Phi_{k}}) \prod_{g,h} \! q(B_{g,h} | \Delta_{B_{\eta}}, \Delta_{B_{\kappa}})
\end{align}

The lower bound for the data log-likelihood from jensen's inequality is: 

\begin{align}
L_{\Delta} &= E_{q}\bigg[ \log \! P(Y, W, z_{\rightarrow}, z_{\leftarrow}, \Phi,
\Pi, B | \kappa, \eta, \beta, \alpha) - \log \! q \bigg]
\end{align}

\begin{eqnarray}
L_{\Delta} = E_{q} \left[ \sum_{t} \! \sum_{p,q} \! \log \bigg(
B_{g,h}^{Y_{p,q}^t} \frac{e^{-B_{gh}}}{Y_{pq}^{t}!} \bigg) +
\sum_{t} \! \sum_{pq} \! \log\bigg( \prod_{k} (\pi_{p,k}^{z_{p \rightarrow q} =
k}) \bigg) + \sum_{t} \! \sum_{p,q} \log \! \bigg( \prod_{k}(\pi_{q,k})^{z_{p
\leftarrow q} = k} \bigg) ~+ \nonumber\\
 \sum_{p} \! \log \left[ \prod_{k}
(\Pi_{p,k})^{\alpha_{k} - 1} \cdot \frac{\Gamma(\sum \alpha_{k})}{\prod_{k}
\Gamma(\alpha_{k})} \right] +
\sum_{t} \! \sum_{p} \! \sum_{w\in R_p^t}  \log \! \bigg(
\prod_{u\in V}(\bar{z}^T\phi_u)^{w = u} \bigg) + \nonumber\\
 \sum_{k} \! \log\left[ \prod_{u\in V}
(\phi_{k,u})^{\beta_{k} - 1} \cdot \frac{\Gamma(\sum \beta_{k})}{\prod_{k}
\Gamma(\beta_{k})} \right] +
 \sum_{g,h} \! \log \! \left( B_{g,h}^{\kappa - 1} /
\eta^{\kappa} \Gamma(\kappa) \cdot \exp(-B_{g,h}/\eta) \right) \right] 
\nonumber
\end{eqnarray}
\begin{eqnarray}
 -E_{q} \left[ \sum_{t} \! \sum_{p,q} \log \big( \prod_{k} (\Delta_{z_{p
\rightarrow q}, k})^{z_{p \rightarrow q}=k} \big) + \sum_{t} \! \sum_{p,q}
\! \log \! \big(
\prod_{k} \! (\Delta_{z_{p \leftarrow q}, k})^{z_{p \leftarrow q} = k} \big)
 + \nonumber \\
 \sum \! \log \big[ \prod_{k} \! (\Pi_{p,k})^{\Delta_{\pi_{pk}}-1}
\frac{\Gamma(\Delta_{\Pi_{p}})}{\prod_{k=1} \! \Gamma(\Delta_{\Pi_{p,k}})} \big]
 + 
\sum_{k} \log \! \big[ \prod_{u \in v}
(\Phi_{k,u})^{\Delta_{\Phi_{ku}} - 1)} \frac{\Gamma(\Delta_{\Phi_{k}})}
{\prod_{u \in v} \! \Gamma(\Delta_{\Phi_{k,u}})} \big] + \nonumber \\ 
\sum_{g,h} \log \! \big[
\frac{B_{g,h}^{\Delta_{\kappa = 1}}}{\Delta_{\eta}^{\Delta_{\kappa}}
\Gamma(\Delta_{\kappa})} \exp(-B_{g,h}/\Delta_{\eta}) \big] \right]
\label{eqn:VarLowerBound}
\end{eqnarray}

Equation~\ref{eqn:VarLowerBound} is the variational lower bound of the log
likelihood function which is to be maximized.
There are terms like $ E_q\left[\sum_{g,h} \! \log \! \left( B_{g,h}^{\kappa -
1} / \eta^{\kappa} \Gamma(\kappa) \cdot \exp(-B_{g,h}/\eta) \right) \right]$
which can be obtained by taking derivation of the partition function of the exponential
family form of gamma distribution. I still have to figure out an effective way
to evaluate $E_q\left[ \sum_{t} \! \sum_{p} \! \sum_{w\in R_p^t}  \log \! \bigg(
\prod_{u\in V}(\bar{z}^T\phi_u)^{w = u} \bigg)\right]$ which Chong suggested
(and Eric too in the last meeting) to evaluate by intriducing an additional
latent variable $\bar{z}_p$ which is a realization of the average $\frac{\sum_{q\in Q} z_{p\rightarrow q}}{|Q|}$. 


\section{Datasets}
We analyse three different forums: 1) Cancer forum, 2) Stack Overflow, and
3) Reddit.
The three datasets mentioned above represent three different sets of
online gatherings which helps us genearlize pur claims. 
% \subsection{Wikipedia talk pages}
% Wikipedia currently hosts more than four million articles on a wide range of topics.
% Quality control on Wikipedia occurs through discussions on the Wikipedia talk pages. 
% Every article on Wikipedia has a corresponding talk-page. Contributors to Wikipedia 
% discuss edits by other users, topics that can be used to extend the article, 
% the veracity of the article's contents etc. Talk-pages provide functionality
% for threaded discussions that are used as dialog among users. This rich
% structured discussion manifests itself as a social network that can be mined and
% studied.  A standard Wikipedia talk page consists of topics which hold
% discussion threads. For building our dataset, we used a snapshot of Wikipedia on
% the 1st of October 2012 \cite{wikipedia}. We built a parser and extracted the
% thread structure in the talk-pages to build the matrices. There are 20,000 users
% in our datasets that span accross 30,000 talk pages ~(\comment{These figures
% will change depending on whether we want to incorporate more or less users in
% future}).
% The talk pages become the threads in context of our graphical model.

\subsection{Cancer Forum}
The cancer forum is an online forum where users discuss about their cancer
treatment and any thing else under the sun. Here again the conversation happens
in a structured way where users post their responses on a thread by thread
basis. Users also call each other by their names (or nick-names) while posting
in many cases. This forum has around 3000 users and 10,000 threads, and a user
on average posts around 120 words in a post.

\subsection{Stack Overflow}
This is an online forum where users ask and answer technical troubles. It is a
typical online forum where users reply to each other in a threaded structure.
Based on the response the replies are voted up and down by other users. This
voting score is used in our prediction tasks later. \comment{Shriphani will
provide the exact statistics of this dataset as soon as the crawl is done. 
We will have most likely have around 1/2 a million users in this set.}


\subsection{Reddit}
Reddit is an online trend spotting website where users post interesting
articles, news, stories, links etc. and a discussion ensues. Users can upvote or
downvote any reply or posts. The converstaions happen in a threaded structure as
described in our generative story. The upvotes are used later by our model for
prediction task. \comment{This is again an ongoing crawl but
we should have atleast 200K+ users here. Shriphani please provide the latest
numbers soon}

% \comment{Need to expand more in the dataset section with more numbers and
% stats. Though the stats would be much clearer as and when we perform the
% experiments}

% 
% \section{Our Method \& Tasks Finished}
% Our approach is to model social emergence via designing a mixed membership
stochastic block model when we have a tensor graph of user-user interaction 
instead of just a single user$\times$user matrix as is the case for the
MMSB~\cite{Airoldi:2008:MMS:1390681.1442798}. We define a tensor 
of user$\times$user$\times$interactions tensor $Y_{N,N,I}$, where $N$ is 
the number of users and $I$ is the number of interaction types. The tensor 
$Y$ is the data observed.

\paragraph{Tasks finished.} We have designed a model that can handle a multitude
of interaction types in a network. The model also takes into account non-binary 
edge-weights in the graph. Our present model is a mixed membership
stochastic model with tensor interactions. The individual interaction between the 
users are drawn form a Poisson distribution.
% We have plans to extend this model to
% take into account the text that users write while interacting with each other.
The generative story of the present model is:  
\begin{itemize}
  \item For each user $p \in \mathcal{N}$
  \begin{itemize}
    \item Draw a $K$ dimensional mixed membership vector 
    $\overset{\rightarrow}{\uppi}_{p} \sim$ Dirichlet($\alpha$)

    \item Draw $B_t(g,h) \sim Gamma(\kappa,\eta)$
  \end{itemize}

  \item For each pair of users $(p, q) \in \mathcal{N} \times \mathcal{N}$:
  \begin{itemize}
    \item Draw membership indicator for the indicator, 
    $\overset{\rightarrow}{z}_{p \rightarrow q} \sim$ Multinomial($\uppi_{p}$).
    \item Draw membership indicator for the receiver, $\overset{\rightarrow}{z}_{q
    \rightarrow p} \sim$ Multinomial($\uppi_{q}$).
    \item Sample the value of their $i-th$ interaction, $Y(p,q,i) \sim$
    Poisson(${\overset{\rightarrow}{z}}^{\top}_{p \rightarrow q}
    B_{i}\overset{\rightarrow}{z}_{p \leftarrow q}$). We make the assumption that
    two interactions are independent of each other i.e $Y(p,q,i)$ and $Y(p,q,j)$
    are independent of each other where $i\neq j$.
    \item Draw $z_{u,m}$ topic for user $U$'s document from $\pi_u$.
    \item Draw $\tau_{k}$ from $Dirichlet(\beta)$.
    \item Draw a word $w_{u,m}$ from $\tau_{z}$
  \end{itemize}
\end{itemize}  

The graphical model for this gnerative scheme is shown in figure 1.

\begin{figure}
\begin{center}
\includegraphics[width= 0.3\textwidth]{non_lda}
\label{fig:gmodel}
\caption{This graphical model shows the generative scheme of our approach. T is the number of interaction types}
\end{center}
\end{figure}



\paragraph{Intuition behind the approach.} The model rests its case on the intuition that multiple context of 
interaction with the ability to model non-binary edge weights (which in the case of wikipedia are counts thus 
using Poisson) should provide better cluster. We are able to leverage multiple dimesions of the data and the 
chances of getting finer clusters are much more here. 

The joint likelihood of this model is:
\begin{equation}
p(Y, \overset{\rightarrow}{\uppi}_{1:N}, Z_{\rightarrow},
Z_{\leftarrow}|\overset{\leftarrow}{\upalpha}, B) =
\prod_{p,q}(\prod_{i}{P(Y(p,q,i)|\overset{\rightarrow}{z}_{p \rightarrow q},
\overset{\rightarrow}{z}_{p \leftarrow q}, B_{i}))P(\overset{\rightarrow} {z}_{p
\rightarrow q}|\overset{\rightarrow}{\uppi}_{p})\prod_{p}
{P(\overset{\rightarrow}{\uppi}_{p}|\overset{\rightarrow}{\upalpha})}}
\end{equation}  
where $B$ is the block tensor and $B_i$ represent the block matrix for $i-th$
interaction. The marginal data likelihood for this model is:

\begin{eqnarray}
&&p(Y|\overset{\rightarrow}{\upalpha}, B) = \nonumber\\
&&\int_{\Pi}{\sum_{Z}{\bigg( \prod_{p,q}(\prod_{i}{P(Y(p,q,i)|
\overset{\rightarrow}{z}_{p \rightarrow q}, 
\overset{\rightarrow}{z}_{p \leftarrow q}, B_i)) P(\overset{\rightarrow}{z}_{p
\rightarrow q} | \overset{\rightarrow}{\uppi}_{q}) P(\overset{\rightarrow}{z}_{p
\leftarrow q}|\overset{\rightarrow}{\uppi}_{q})\prod_{p}{P(\overset{
\rightarrow}{\uppi}_{p}|\overset{\rightarrow}{\upalpha})}} \bigg)
d\overset{\rightarrow}{\uppi}}} \nonumber\\
\end{eqnarray} 

For results presented here we use collapsed Gibbs sampling for parameter estimation, though 
we also derived variational updates for our model described in \ref{sec:app-1}.
In this section we report the sampling updates steps. We collapse latent variables 
$\overset{\rightarrow}{z}_{p \rightarrow q}$ and $\overset{\rightarrow}{z}_{p \leftarrow q}$ 
together. 
For Poisson case we have:
 \begin{eqnarray}
    p(z|\alpha) &= \int \! p(z|\phi) \cdot p(\phi|\alpha) \, \mathrm{d}\phi\\ \nonumber
    &= \prod_{p} \int \! {\prod_{q}{\prod_{i}^{k}{\pi_{p,i}^{z_{p \rightarrow q},i}}}  \frac{\prod_{i}^{k}{\pi_{p,i}^{\alpha_{i}}}}{Dir(\alpha_{i})} } \, \mathrm{d} \pi_{p}\\ \nonumber
    &= \prod_{p} \frac{\int \! {\prod_{i}^{k}{\pi_{p,i}^{n_{p_{i}} \rightarrow \alpha_{i}}}} \, \mathrm{d} \pi_{p}}{Dir(\alpha+{i})}\\ \nonumber
    &= \bigg[ \frac{Dir(n_{p}^{i} + \alpha_{i}, \dots)}{Dir(\alpha_{i})} \bigg]
  \end{eqnarray}

where $n_{p}^{i}$ is the number of times user $p$ is assigned cluster $i$. And 

\begin{eqnarray}
    P(Y|\eta, z, \kappa) = \prod_{t=1}^{T} \! \prod_{g,h} \left[ \frac{\Gamma(\sum_{p \in g, q \in h} Y_t(p,q) + \kappa) \cdot (n_{t,g,h} + \frac{1}{\eta})^{-(\sum_{p \in g, q \in h} Y_{t}(p,q) + \kappa)}}{\prod_{p \in g, q \in h} Y_t(p,q)! \cdot \eta^{\kappa} \Gamma(\kappa)} \right]\\ \nonumber
  \end{eqnarray}

where $n_{t,g,h}$ is the number of times cluster pair $g,h$ is picked in the interaction context $t$.
Using the above 2 equations the sampling updates are:
\begin{eqnarray}
    & &p(z_{p \rightarrow q = g}, z_{q \rightarrow p = h} | z_{\neg (p \rightarrow q, p \leftarrow q)}, Y, \eta, \alpha, \kappa)
    =\\ \nonumber & &\prod_{t=1}^{T} \left[  \frac{\prod_{i=1}^{Y_t(p \rightarrow q=g,p \leftarrow q=h)} \! (\sum_{a\in g, b \in h} Y_t(a,b) + \kappa - i)}{Y_t(p,q)} \right] \cdot\\ \nonumber
    & & \prod_{t=1}^{T}\left[\frac{(n_{t,g,h} + \frac{1}{\eta})^{-[\sum_{p \in g, q \in h} Y_{t} (p,q) + k]}}{\left[ \left(n_{t,g,h} + \frac{1}{\eta}^{-(\sum_{p \in g, q \in h} Y_{t}(p,q) + \kappa)} \right) \right]_{\neg (p \rightarrow \leftarrow q)}} \right]  \cdot \\ \nonumber
 & & \frac{(n_p^g + \alpha_g)^{\neg (p \rightarrow q = g)}}{(n_p + \alpha)^{(\neg p \rightarrow q =g)}} \frac{(n_q^h + \alpha)^{\neg (p \leftarrow q = h)}}{(n_q + \alpha)^{(\neg p \leftarrow q =h)}}
\end{eqnarray}

The parameter estimation of $\Pi$ and $B$ is:
\begin{eqnarray}
p(B_t(g,h) | \overset{\rightarrow}{z}, \eta, \overset{\rightarrow}{Y_t}) &= (\sum_{p \in g, q \in h} Y_t(p,q) + \kappa) \cdot (n_{g,h} + \frac{1}{\eta})^{-1}
\end{eqnarray}

\begin{eqnarray}
\pi_p^i &= \frac{n_p^i + \alpha^i}{\sum_{i=1}^{k} (n_p^i + \alpha^{i})}
\end{eqnarray}

This above equations for sampling updates and parameter estimation are derived in detail in appendix ~\ref{sec:app-2}.






% 
\section{Experiments \& Results \& Evaluations}
\comment{
% Following are the experimemts that we are planning to do\\
% 1. Analyse and interpret the communties discovered on Wikipedia talk pages and
% Cancer forum providing interesting insights and telling how modelling forum
% structure helps. \\
% 2. Validate the above discovered communities either through held out likelihood
% or perplexity. \\ 
% 3. Introduce a prediction task; there were several suggestions: \\
% \hspace{3 cm} a) predict the topic of the posts by a user in a thread where for
% training we have hand-labeled thread posts (suggested by Chong) \\
% \hspace{3 cm} b) Hold out some users on some threads and predict whether the
% user is going to post on the held-out threads \\
% 3. Predicting the popularity of a response/post on Reddit as well as Stack
% Overflow. We can go much finer and predict whether a given user will upvote a
% post or not. We will also have to derive equations for this but shouldn't be
% too hard.
}



\section{Conclusion \& Future Work}
\comment{1) Currently our model just picks up one signal for the network
component (MMSB part) i.e. in other words we are just modelling one type of
interaction or just one graph, but as we did for the SEI model (tensor model) 
there are multiple types of
networks/graphs/interactions. Future work can incorporate this.\\
2) Adding temporal dimension to this model would be a very interesting idea.
E.g. how threads evolve over time, or how user behavior changes, or how new
communities emerge in the forum etc.}

%o\section{Finished Task Split-up}

%\label{sec:appendix}
% \onecolumn{}

%The log-likelihood of the model:
%\begin{align}
%\log L &= \log \! P(Y, W, Z_{\leftarrow}, 
%Z_{\rightarrow}, \Pi, B, \beta | \alpha, \eta, \theta, \alpha) \nonumber\\
%\nonumber &= \sum_{t} \bigg[ \sum_{p,q} \! \log P(Y_{t,p, q} | Z_{t,p \rightarrow q} 
%     , Z_{t,p \leftarrow q}, B) \\ \nonumber 
%     &+ \sum_{p,q} \log P(Z_{t, p \rightarrow q} | \Pi_q) \\ \nonumber 
%     & + \sum_{p,q} \log \! P(Z_{t, p \leftarrow q} | \Pi_{q}) \bigg] 
%     + \sum_{p} \log \! P(\Pi_{p} | \alpha)  \\ \nonumber 
%     & + \bigg[ \sum_{t=1}^{T} \! \sum_{p \in t} \sum_{i=1}^{N_{T_{p}}} 
%     \log \! P(w_{t,p,i} | Z'_{t,p,i}, \beta) \\ \nonumber 
%     & + \sum_{t=1}^{T} \sum_{p \in t} \sum_{i=1^{N_{T_{p}}}} \log \! 
%     P(Z'_{t,p,i} | \bar{Z}_{t, p \rightarrow q}) \bigg]   
%     \\ \nonumber & + \sum_{k} \log P(\beta_{k} | 
%     \eta) +  \sum_{g,h} \log P(B_{g,h} | \kappa, \theta).\\ 
%     \label{eqn:LL}
%\end{align}

%The data likelihood for the model in figure~1
%
%\begin{align}
%P(Y, R_{p} | \alpha, \beta, \kappa, \eta) &=  \nonumber\\ 
% \int_{\Phi} \!
%\int_{\Pi} \sum_{z} \! P(Y, R_{p}, & z_{p \rightarrow q}, z_{p \leftarrow q},
%\Phi, \Pi | \alpha, \beta, \kappa, \eta)  \nonumber \\  \nonumber
%%\\ 
%= \int_{\Phi} \! \int_{\Pi} \sum_{z} \! \bigg[ \prod_{p,q} & \prod_{t}
%P(Y_{pq}^{t} | z_{p \rightarrow q}^{t}, z_{p \leftarrow q}^{t}, B) 
%\cdot P(z_{p \rightarrow q}^{t} | \Pi_{p}) \nonumber
%\\  \cdot P(z_{p \leftarrow q}^{t} |
%\Pi_{q})   & \cdot \bigg(\prod_{p} P(\Pi_{p} | \alpha) \prod_{t} \prod_{p}
%P(R_{p}^{t} | z_{p \rightarrow q}^{t}, \Phi) \nonumber
%\\ \cdot \prod_{k} P(\Phi_{k} |
%\beta)&\bigg) \cdot \prod_{g,h}P(B_{gh} | \eta, \kappa) \bigg].
%\end{align}

%The complete log likelihood of the model is:
%
%\begin{align}
%\log \! &P(Y, W, z_{\rightarrow}, z_{\leftarrow}, \Phi, \Pi, B | \kappa, \eta,
%\beta, \alpha) \nonumber
%\\ & = \sum_{t} \! \sum_{p,q} \! \log P(Y_{pq}^{t} | z_{p
%\rightarrow q}^{t} , z_{p \leftarrow q}^{t}, B) \nonumber
%\\ &+ \nonumber \sum_{t} \!
%\sum_{p,q} \! (\log P(z_{p \rightarrow q}^{t} | \Pi_{p}) + \log \! P(z_{p \leftarrow q}^{t} |
%\Pi_{p})) \\  
%&+ \sum_{p} \! \log \! P(\Pi_{p} | \alpha) ~+ \sum_{t} \!
%\sum_{p} \! \sum_{w \in R_{p}^{t}} \log P(w | z_{p \rightarrow}, \Phi)
%\nonumber\\ 
%&+ \sum_{k} \! \log P(\Phi_{k} | \beta) + \sum_{gh} \! \log P(B_{gh} | \eta,
%\kappa).
%\end{align}

%The mean field variational approximation for the posterior is 
%
%\begin{align}
%q(z, &\Phi, \Pi, B | \Delta_{z_{\rightarrow}}, \Delta_{\Phi}, \Delta_{B},
%\Delta_{z_{\leftarrow}}, \Delta_{B_{\kappa}})  = \nonumber \\ \prod_{t} \!
%& \prod_{p,q} \! \bigg( q_{1}(z_{p \rightarrow q}^{t} | \Delta_{z_{p \rightarrow
%q}}) + q_{1}(z_{p \leftarrow q}^{t} | \Delta_{z_{p \leftarrow q}})  \bigg) \nonumber \\
%\cdot \prod_{p} &\! q_{4}(\Pi_{p} | \Delta_{\Pi_{p}}) \prod_{k} q_{3} (\Phi_{k}
%| \Delta_{\Phi_{k}}) \prod_{g,h} \! q(B_{g,h} | \Delta_{B_{\eta}},
%\Delta_{B_{\kappa}}).
%\end{align}

The lower bound for the data log-likelihood from jensen's inequality is: 

\begin{align}
&L_{\Delta} = E_{q}\bigg[ \log \! P(Y, W, z_{\rightarrow}, z_{\leftarrow}, \Phi,
\Pi, B | \kappa, \eta, \beta, \alpha) - \log \! q \bigg]\nonumber\\
&= E_{q} \Bigg[ \sum_{t} \! \sum_{p,q} \! \log \left(
B_{g,h}^{Y_{p,q}^t} \frac{e^{-B_{gh}}}{Y_{pq}^{t}!} \right) +
\sum_{t} \! \sum_{pq} \! \log\left( \prod_{k} (\pi_{p,k}^{z_{p \rightarrow q} =
k}) \right) \nonumber\\
&+ \sum_{t} \! \sum_{p,q} \log \! \left(
\prod_{k}(\pi_{q,k})^{z_{p \leftarrow q} = k} \right)\nonumber\\ 
&+\sum_{t} \! \sum_{p} \! \sum_{w\in R_p^t}  \log \! \left(
\prod_{u\in V}(\bar{z}^T\phi_u)^{w = u} \right)
\nonumber\\ &+ 
\sum_{p} \! \log \left( \prod_{k}
(\Pi_{p,k})^{\alpha_{k} - 1} \cdot \frac{\Gamma(\sum \alpha_{k})}{\prod_{k}
\Gamma(\alpha_{k})} \right) \nonumber\\ & + 
\sum_{k} \! \log\left( \prod_{u\in V}
(\phi_{k,u})^{\beta_{k} - 1} \cdot \frac{\Gamma(\sum \beta_{k})}{\prod_{k}
\Gamma(\beta_{k})} \right) \nonumber\\ &+
 \sum_{g,h} \! \log \! \left( B_{g,h}^{\kappa - 1} /
\eta^{\kappa} \Gamma(\kappa) \cdot \exp(-B_{g,h}/\eta) \right) \Bigg]
\nonumber\\ 
& -E_{q} \Bigg[ \sum_{t} \! \sum_{p,q} \log \big( \prod_{k} (\Delta_{z_{p
\rightarrow q}, k})^{z_{p \rightarrow q}=k} \big) \nonumber \\&+ \sum_{t} \!
\sum_{p,q} \! \log \! \left(
\prod_{k} \! (\Delta_{z_{p \leftarrow q}, k})^{z_{p \leftarrow q} = k} \right)
  \nonumber \\
 &+\sum \! \log \left( \prod_{k} \! (\Pi_{p,k})^{\Delta_{\pi_{pk}}-1}
\frac{\Gamma(\Delta_{\Pi_{p}})}{\prod_{k=1} \! \Gamma(\Delta_{\Pi_{p,k}})}
\right) \nonumber \\ &+ 
\sum_{k} \log \! \left( \prod_{u \in v}
(\Phi_{k,u})^{\Delta_{\Phi_{ku}} - 1)} \frac{\Gamma(\Delta_{\Phi_{k}})}
{\prod_{u \in v} \! \Gamma(\Delta_{\Phi_{k,u}})} \right) \nonumber \\ 
&+ \sum_{g,h} \log \! \left(
\frac{B_{g,h}^{\Delta_{\kappa = 1}}}{\Delta_{\eta}^{\Delta_{\kappa}}
\Gamma(\Delta_{\kappa})} \exp(-B_{g,h}/\Delta_{\eta}) \right) \Bigg].
\label{eqn:VarLowerBound}
\end{align}

%$\Delta_{\phi}$ used in the update of $\phi$ in equation~\ref{eqn:phiUp}. The
%parameter $\omega$ is used here to balance out the contribution from the text
%side to the network. 
%
%\begin{align}
%\Delta_{\phi^{'}_{t,p,g,h}} &= y_{t,p,q}( \log \! \lambda_{g,h} + 
%\Psi(\nu_{g,h})) - \nu_{g,h} \lambda_{g,h} - \log \! (y_{t,p,q}!)
%\nonumber \\ & + \Psi(\gamma_{p,g}) - \Psi(\sum_{g} \gamma_{p,g})
%\nonumber \\  & + \Psi(\gamma_{q,h}) - \Psi(\sum_{h} \gamma_{q,h})
%\nonumber \\  & + \omega\sum_{i=1}^{N_{T_{P}}} \! \chi_{t,p,i,g} 
%\bigg[ \ln \! \frac{\epsilon}{\delta_{p,t}} - \frac{1}{\delta_{t,p}} 
%+ \ln \bigg( 1 + \frac{\epsilon}{\delta_{p,t}} \bigg) 
%\cdot \frac{1}{\delta_{t,p}} \bigg].
%\label{eqn:phiDelta}
%\end{align}
%
%$\Delta_{\chi}$ used in equation~\ref{eqn:chiUp} for $\chi$ update
%
%\begin{align}
%\Delta_{\chi^{'}_{t,p,g,h}} &= \bigg[ \Psi(\tau_{k,w_{t,p,i}}) - 
%\Psi(\sum_{w_{t,p,i}} \! \tau_{k, w_{t,p,i}}) \bigg]
%\nonumber \\  & + \ln \! (\frac{\epsilon}{\delta{p,t}}) \frac{1 - 
%\sum_{q,h} \! \phi_{t,p,q,k,j}}{\delta_{t,p}} 
%\nonumber\\  & + \frac{\sum_{q,h} \phi_{t,p,q,k,h}}{\delta_{t,p}} 
%\ln \! (1 + \frac{\epsilon}{\delta_{t,p}}).
%%\nonumber\\  & - 1 - \log \! \chi_{t,p,i,k}
%\label{eqn:chiDelta}
%\end{align}


Partial derivative of $\nu$
\begin{align}
\frac{dL}{\partial\nu_{g,h}} &= \sum_{t} \! \sum_{p,q} \! 
\phi_{t,p,q,g,h} (y_{t,p,q} \Psi'(\nu_{g,h}) - \lambda_{g,h})
\nonumber\\  & + (\kappa_{g,h} - \nu_{g,h})\Psi'(\nu_{g,h}) + 
1 - \frac{\lambda_{g,h}}{\theta_{g,h}}.
\label{eqn:partialNu}
\end{align}

The traditional variational updates for the global parameters

\begin{align}
\gamma_{p,k} &= \alpha_{k} + \sum_{t} \! \sum_{q} \! \sum_{h} \! \phi_{t,p,q,k,h} 
+ \sum_{t} \! \sum_{q} \! \sum_{g} \! \phi_{t,q,p,g,k}.
\label{eqn:gammaUp}
\end{align}

\begin{align}
\nu_{g,h}^{t+1} &= \nu_{g,h}^{t}+\rho_\nu \frac{dL}{\partial\nu_{g,h}}. 
\label{eqn:nuUp}
\end{align}

\begin{align}
\lambda_{g,h} &= \frac{\bigg( \sum_{t} \! \sum_{p,q} \! \phi_{t,p,q,g,h} y_{t,p,q} + 
\kappa_{g,h} \bigg) }{
 \bigg( \bigg( \sum_{t} \! \sum_{p,q} \! \phi_{t,p,q,g,h} \bigg) + 
\frac{1}{\theta_{g,h}} \bigg) \nu_{g,h}}.
\label{eqn:lambdaUp}
\end{align}

\begin{align}
\tau_{p,v} = \nu_{v} + \sum_{t} \! \sum_{p \in t}
\bigg(\sum_{w_{t,p,i}=v}^{N_{t,p}} \chi_{t,p,i,k} \bigg).
\label{eqn:tauUp}
\end{align}


where $\rho_\nu$ is $\nu$'s gradient ascent update step-size using its partial
derivative $\frac{dL}{\partial\nu_{g,h}}$ define in equation~\ref{eqn:partialNu}.



%\input{06Eval01}
%\input{06Eval02}
\bibliography{mybib}
\bibliographystyle{plain}
\end{document}
