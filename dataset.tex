We analyse two different forums: 1) Wikipdia talk pages, and 2) Cancer forum.
The two datasets mentioned above represnt two different sets of
online gatherings which helps us genearlize pur claims. 
\subsection{Wikipedia talk pages}
Wikipedia currently hosts more than four million articles on a wide range of topics.
Quality control on Wikipedia occurs through discussions on the Wikipedia talk pages. 
Every article on Wikipedia has a corresponding talk-page. Contributors to Wikipedia 
discuss edits by other users, topics that can be used to extend the article, 
the veracity of the article's contents etc. Talk-pages provide functionality
for threaded discussions that are used as dialog among users. This rich
structured discussion manifests itself as a social network that can be mined and
studied.  A standard Wikipedia talk page consists of topics which hold
discussion threads. For building our dataset, we used a snapshot of Wikipedia on
the 1st of October 2012 \cite{wikipedia}. We built a parser and extracted the
thread structure in the talk-pages to build the matrices. There are 20,000 users
in our datasets that span accross 30,000 talk pages ~(\comment{These figures
will change depending on whether we want to incorporate more or less users in
future}).
The talk pages become the threads in context of our graphical model.

\subsection{Cancer Forum}
The cancer forum is an online forum where users discuss about their cancer
treatment and any thing else under the sun. Here again the conversation happens
in a structured way where users post their responses on a thread by thread
basis. Users also call each other by their names (or nick-names) while posting
in many cases. This forum has around 3000 users and 10,000 threads, and a user
on average posts around 120 words in a post.

\comment{Need to expand more in the dataset section with more numbers and
stats. Though the stats would be much clearer as and when we perform the
experiments}
