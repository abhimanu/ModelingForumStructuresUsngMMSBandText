We discuss here the ideas and questions that are important for modeling
structure in online forums. Following are the questions that we want answered:
\begin{enumerate}
  \item When two persons interact in a thread or a post which topic/community they
each belong to
	\item When a user U is in communnity C what type of text does he use to
	communicate
	\item Multi-user-Interaction: In a thread a user can post by addressing to a
	specific user but he is also talking to other users in the thread simultaneously. Can we model this
	phenomenon
	\item There is an inherent bias towards the thread starter or in turn topic of
	the theread; can such an information be utilised in some form of a prior
	value/input
	\item Multi-layer-Interaction: On the network side of things there are multiple
	signals which cannt be simply added to make a single signal e.g. different
	types of edges in the graphs (user calling by username and nick-name). Can the
	model take this into account. We are not doing this at present.
	\item User posts aggregation; there are multiple ways to aggreagte
	\begin{enumerate}
		\item Network Layer aggregation: We call all types of edges as a single
		edge type and use this combined signal. 
		\item aggregating user posts accross multiple threads in the forum.
		\item Aggregating user post only in the same thread
		\item Aggregating user post only for same user-user pair interaction; i.e. a user might have
		posted multiple replies to another user and we aggregate all such replies into
		one for this user pair interaction.
		\item No aggregation at all.
	\end{enumerate}
\end{enumerate}
