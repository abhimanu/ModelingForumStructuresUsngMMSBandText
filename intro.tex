There have been a flood of online forums in recent decade and consequently so
have been the focus of academic research and industry on online social networks. 
Analysing online social networks and user forums have been approached using
various perspective such as graph/network ~\cite{Shi:2000:NCI:351581.351611,
Shi00learningsegmentation} , probabilistic 
graphical model~\cite{ Airoldi:2008:MMS:1390681.1442798}, 
combined network \& text mining
based~\cite{Ho:2012:DHT:2187836.2187936,Nallapati:2008:JLT:1401890.1401957}
based approaches.
But very few of these have taken into account the structural framework in which
the conversation in online forums happen. This is important to correctly model the
interaction as well as the contents posted by the users during their
conversation with the user community. E.g. in an onlne forum there are topic-threads 
and users
post their responses on this thread after possibly reading through the responses
of other users in this thread. And the users possibly posts multiple times on
the thread in the form of replies to other posts in the thread. For analysing
such a user interaction it becomes imperative that the structure of the
conversation must also be taken into account  besides taking into account the 
user interaction
network and the text posted. This enables us to gain
deeper insights into user behavior in the online community that was not possible
earlier. Very few research works have tried to bring the forum structure in the
analysis of online communities. This is what set our work apart from the past works, our
approach here besides bringing network modeling and text mining together adds in
the forum structure in the model to provide a more robust analysis of
the user interactions. The model also incorporates strength of interaction
among the users by incorporating interaction counts as compared to MMSB model
which just looks presence or absence of link~\cite{Airoldi:2008:MMS:1390681.1442798}. 
In the process we discover interesting online communities and social phenomena.

The current work also focuses on analysing large scale user interactions in big
online social forums. We provide a stochastic variational
approximation~\cite{Hoffman:2013:SVI} based estimation technique that is
scalable to big forums with thousands of users.

% ~\comment{Also write about stochastic approximation if given we have time and we
% can get it working}. 


